
% Default to the notebook output style

    


% Inherit from the specified cell style.




    
\documentclass[11pt]{article}

    
    
    \usepackage[T1]{fontenc}
    % Nicer default font (+ math font) than Computer Modern for most use cases
    \usepackage{mathpazo}

    % Basic figure setup, for now with no caption control since it's done
    % automatically by Pandoc (which extracts ![](path) syntax from Markdown).
    \usepackage{graphicx}
    % We will generate all images so they have a width \maxwidth. This means
    % that they will get their normal width if they fit onto the page, but
    % are scaled down if they would overflow the margins.
    \makeatletter
    \def\maxwidth{\ifdim\Gin@nat@width>\linewidth\linewidth
    \else\Gin@nat@width\fi}
    \makeatother
    \let\Oldincludegraphics\includegraphics
    % Set max figure width to be 80% of text width, for now hardcoded.
    \renewcommand{\includegraphics}[1]{\Oldincludegraphics[width=.8\maxwidth]{#1}}
    % Ensure that by default, figures have no caption (until we provide a
    % proper Figure object with a Caption API and a way to capture that
    % in the conversion process - todo).
    \usepackage{caption}
    \DeclareCaptionLabelFormat{nolabel}{}
    \captionsetup{labelformat=nolabel}

    \usepackage{adjustbox} % Used to constrain images to a maximum size 
    \usepackage{xcolor} % Allow colors to be defined
    \usepackage{enumerate} % Needed for markdown enumerations to work
    \usepackage{geometry} % Used to adjust the document margins
    \usepackage{amsmath} % Equations
    \usepackage{amssymb} % Equations
    \usepackage{textcomp} % defines textquotesingle
    % Hack from http://tex.stackexchange.com/a/47451/13684:
    \AtBeginDocument{%
        \def\PYZsq{\textquotesingle}% Upright quotes in Pygmentized code
    }
    \usepackage{upquote} % Upright quotes for verbatim code
    \usepackage{eurosym} % defines \euro
    \usepackage[mathletters]{ucs} % Extended unicode (utf-8) support
    \usepackage[utf8x]{inputenc} % Allow utf-8 characters in the tex document
    \usepackage{fancyvrb} % verbatim replacement that allows latex
    \usepackage{grffile} % extends the file name processing of package graphics 
                         % to support a larger range 
    % The hyperref package gives us a pdf with properly built
    % internal navigation ('pdf bookmarks' for the table of contents,
    % internal cross-reference links, web links for URLs, etc.)
    \usepackage{hyperref}
    \usepackage{longtable} % longtable support required by pandoc >1.10
    \usepackage{booktabs}  % table support for pandoc > 1.12.2
    \usepackage[inline]{enumitem} % IRkernel/repr support (it uses the enumerate* environment)
    \usepackage[normalem]{ulem} % ulem is needed to support strikethroughs (\sout)
                                % normalem makes italics be italics, not underlines
    

    
    
    % Colors for the hyperref package
    \definecolor{urlcolor}{rgb}{0,.145,.698}
    \definecolor{linkcolor}{rgb}{.71,0.21,0.01}
    \definecolor{citecolor}{rgb}{.12,.54,.11}

    % ANSI colors
    \definecolor{ansi-black}{HTML}{3E424D}
    \definecolor{ansi-black-intense}{HTML}{282C36}
    \definecolor{ansi-red}{HTML}{E75C58}
    \definecolor{ansi-red-intense}{HTML}{B22B31}
    \definecolor{ansi-green}{HTML}{00A250}
    \definecolor{ansi-green-intense}{HTML}{007427}
    \definecolor{ansi-yellow}{HTML}{DDB62B}
    \definecolor{ansi-yellow-intense}{HTML}{B27D12}
    \definecolor{ansi-blue}{HTML}{208FFB}
    \definecolor{ansi-blue-intense}{HTML}{0065CA}
    \definecolor{ansi-magenta}{HTML}{D160C4}
    \definecolor{ansi-magenta-intense}{HTML}{A03196}
    \definecolor{ansi-cyan}{HTML}{60C6C8}
    \definecolor{ansi-cyan-intense}{HTML}{258F8F}
    \definecolor{ansi-white}{HTML}{C5C1B4}
    \definecolor{ansi-white-intense}{HTML}{A1A6B2}

    % commands and environments needed by pandoc snippets
    % extracted from the output of `pandoc -s`
    \providecommand{\tightlist}{%
      \setlength{\itemsep}{0pt}\setlength{\parskip}{0pt}}
    \DefineVerbatimEnvironment{Highlighting}{Verbatim}{commandchars=\\\{\}}
    % Add ',fontsize=\small' for more characters per line
    \newenvironment{Shaded}{}{}
    \newcommand{\KeywordTok}[1]{\textcolor[rgb]{0.00,0.44,0.13}{\textbf{{#1}}}}
    \newcommand{\DataTypeTok}[1]{\textcolor[rgb]{0.56,0.13,0.00}{{#1}}}
    \newcommand{\DecValTok}[1]{\textcolor[rgb]{0.25,0.63,0.44}{{#1}}}
    \newcommand{\BaseNTok}[1]{\textcolor[rgb]{0.25,0.63,0.44}{{#1}}}
    \newcommand{\FloatTok}[1]{\textcolor[rgb]{0.25,0.63,0.44}{{#1}}}
    \newcommand{\CharTok}[1]{\textcolor[rgb]{0.25,0.44,0.63}{{#1}}}
    \newcommand{\StringTok}[1]{\textcolor[rgb]{0.25,0.44,0.63}{{#1}}}
    \newcommand{\CommentTok}[1]{\textcolor[rgb]{0.38,0.63,0.69}{\textit{{#1}}}}
    \newcommand{\OtherTok}[1]{\textcolor[rgb]{0.00,0.44,0.13}{{#1}}}
    \newcommand{\AlertTok}[1]{\textcolor[rgb]{1.00,0.00,0.00}{\textbf{{#1}}}}
    \newcommand{\FunctionTok}[1]{\textcolor[rgb]{0.02,0.16,0.49}{{#1}}}
    \newcommand{\RegionMarkerTok}[1]{{#1}}
    \newcommand{\ErrorTok}[1]{\textcolor[rgb]{1.00,0.00,0.00}{\textbf{{#1}}}}
    \newcommand{\NormalTok}[1]{{#1}}
    
    % Additional commands for more recent versions of Pandoc
    \newcommand{\ConstantTok}[1]{\textcolor[rgb]{0.53,0.00,0.00}{{#1}}}
    \newcommand{\SpecialCharTok}[1]{\textcolor[rgb]{0.25,0.44,0.63}{{#1}}}
    \newcommand{\VerbatimStringTok}[1]{\textcolor[rgb]{0.25,0.44,0.63}{{#1}}}
    \newcommand{\SpecialStringTok}[1]{\textcolor[rgb]{0.73,0.40,0.53}{{#1}}}
    \newcommand{\ImportTok}[1]{{#1}}
    \newcommand{\DocumentationTok}[1]{\textcolor[rgb]{0.73,0.13,0.13}{\textit{{#1}}}}
    \newcommand{\AnnotationTok}[1]{\textcolor[rgb]{0.38,0.63,0.69}{\textbf{\textit{{#1}}}}}
    \newcommand{\CommentVarTok}[1]{\textcolor[rgb]{0.38,0.63,0.69}{\textbf{\textit{{#1}}}}}
    \newcommand{\VariableTok}[1]{\textcolor[rgb]{0.10,0.09,0.49}{{#1}}}
    \newcommand{\ControlFlowTok}[1]{\textcolor[rgb]{0.00,0.44,0.13}{\textbf{{#1}}}}
    \newcommand{\OperatorTok}[1]{\textcolor[rgb]{0.40,0.40,0.40}{{#1}}}
    \newcommand{\BuiltInTok}[1]{{#1}}
    \newcommand{\ExtensionTok}[1]{{#1}}
    \newcommand{\PreprocessorTok}[1]{\textcolor[rgb]{0.74,0.48,0.00}{{#1}}}
    \newcommand{\AttributeTok}[1]{\textcolor[rgb]{0.49,0.56,0.16}{{#1}}}
    \newcommand{\InformationTok}[1]{\textcolor[rgb]{0.38,0.63,0.69}{\textbf{\textit{{#1}}}}}
    \newcommand{\WarningTok}[1]{\textcolor[rgb]{0.38,0.63,0.69}{\textbf{\textit{{#1}}}}}
    
    
    % Define a nice break command that doesn't care if a line doesn't already
    % exist.
    \def\br{\hspace*{\fill} \\* }
    % Math Jax compatability definitions
    \def\gt{>}
    \def\lt{<}
    % Document parameters
    \title{04\_preprocessing\_and\_training}
    
    
    

    % Pygments definitions
    
\makeatletter
\def\PY@reset{\let\PY@it=\relax \let\PY@bf=\relax%
    \let\PY@ul=\relax \let\PY@tc=\relax%
    \let\PY@bc=\relax \let\PY@ff=\relax}
\def\PY@tok#1{\csname PY@tok@#1\endcsname}
\def\PY@toks#1+{\ifx\relax#1\empty\else%
    \PY@tok{#1}\expandafter\PY@toks\fi}
\def\PY@do#1{\PY@bc{\PY@tc{\PY@ul{%
    \PY@it{\PY@bf{\PY@ff{#1}}}}}}}
\def\PY#1#2{\PY@reset\PY@toks#1+\relax+\PY@do{#2}}

\expandafter\def\csname PY@tok@w\endcsname{\def\PY@tc##1{\textcolor[rgb]{0.73,0.73,0.73}{##1}}}
\expandafter\def\csname PY@tok@c\endcsname{\let\PY@it=\textit\def\PY@tc##1{\textcolor[rgb]{0.25,0.50,0.50}{##1}}}
\expandafter\def\csname PY@tok@cp\endcsname{\def\PY@tc##1{\textcolor[rgb]{0.74,0.48,0.00}{##1}}}
\expandafter\def\csname PY@tok@k\endcsname{\let\PY@bf=\textbf\def\PY@tc##1{\textcolor[rgb]{0.00,0.50,0.00}{##1}}}
\expandafter\def\csname PY@tok@kp\endcsname{\def\PY@tc##1{\textcolor[rgb]{0.00,0.50,0.00}{##1}}}
\expandafter\def\csname PY@tok@kt\endcsname{\def\PY@tc##1{\textcolor[rgb]{0.69,0.00,0.25}{##1}}}
\expandafter\def\csname PY@tok@o\endcsname{\def\PY@tc##1{\textcolor[rgb]{0.40,0.40,0.40}{##1}}}
\expandafter\def\csname PY@tok@ow\endcsname{\let\PY@bf=\textbf\def\PY@tc##1{\textcolor[rgb]{0.67,0.13,1.00}{##1}}}
\expandafter\def\csname PY@tok@nb\endcsname{\def\PY@tc##1{\textcolor[rgb]{0.00,0.50,0.00}{##1}}}
\expandafter\def\csname PY@tok@nf\endcsname{\def\PY@tc##1{\textcolor[rgb]{0.00,0.00,1.00}{##1}}}
\expandafter\def\csname PY@tok@nc\endcsname{\let\PY@bf=\textbf\def\PY@tc##1{\textcolor[rgb]{0.00,0.00,1.00}{##1}}}
\expandafter\def\csname PY@tok@nn\endcsname{\let\PY@bf=\textbf\def\PY@tc##1{\textcolor[rgb]{0.00,0.00,1.00}{##1}}}
\expandafter\def\csname PY@tok@ne\endcsname{\let\PY@bf=\textbf\def\PY@tc##1{\textcolor[rgb]{0.82,0.25,0.23}{##1}}}
\expandafter\def\csname PY@tok@nv\endcsname{\def\PY@tc##1{\textcolor[rgb]{0.10,0.09,0.49}{##1}}}
\expandafter\def\csname PY@tok@no\endcsname{\def\PY@tc##1{\textcolor[rgb]{0.53,0.00,0.00}{##1}}}
\expandafter\def\csname PY@tok@nl\endcsname{\def\PY@tc##1{\textcolor[rgb]{0.63,0.63,0.00}{##1}}}
\expandafter\def\csname PY@tok@ni\endcsname{\let\PY@bf=\textbf\def\PY@tc##1{\textcolor[rgb]{0.60,0.60,0.60}{##1}}}
\expandafter\def\csname PY@tok@na\endcsname{\def\PY@tc##1{\textcolor[rgb]{0.49,0.56,0.16}{##1}}}
\expandafter\def\csname PY@tok@nt\endcsname{\let\PY@bf=\textbf\def\PY@tc##1{\textcolor[rgb]{0.00,0.50,0.00}{##1}}}
\expandafter\def\csname PY@tok@nd\endcsname{\def\PY@tc##1{\textcolor[rgb]{0.67,0.13,1.00}{##1}}}
\expandafter\def\csname PY@tok@s\endcsname{\def\PY@tc##1{\textcolor[rgb]{0.73,0.13,0.13}{##1}}}
\expandafter\def\csname PY@tok@sd\endcsname{\let\PY@it=\textit\def\PY@tc##1{\textcolor[rgb]{0.73,0.13,0.13}{##1}}}
\expandafter\def\csname PY@tok@si\endcsname{\let\PY@bf=\textbf\def\PY@tc##1{\textcolor[rgb]{0.73,0.40,0.53}{##1}}}
\expandafter\def\csname PY@tok@se\endcsname{\let\PY@bf=\textbf\def\PY@tc##1{\textcolor[rgb]{0.73,0.40,0.13}{##1}}}
\expandafter\def\csname PY@tok@sr\endcsname{\def\PY@tc##1{\textcolor[rgb]{0.73,0.40,0.53}{##1}}}
\expandafter\def\csname PY@tok@ss\endcsname{\def\PY@tc##1{\textcolor[rgb]{0.10,0.09,0.49}{##1}}}
\expandafter\def\csname PY@tok@sx\endcsname{\def\PY@tc##1{\textcolor[rgb]{0.00,0.50,0.00}{##1}}}
\expandafter\def\csname PY@tok@m\endcsname{\def\PY@tc##1{\textcolor[rgb]{0.40,0.40,0.40}{##1}}}
\expandafter\def\csname PY@tok@gh\endcsname{\let\PY@bf=\textbf\def\PY@tc##1{\textcolor[rgb]{0.00,0.00,0.50}{##1}}}
\expandafter\def\csname PY@tok@gu\endcsname{\let\PY@bf=\textbf\def\PY@tc##1{\textcolor[rgb]{0.50,0.00,0.50}{##1}}}
\expandafter\def\csname PY@tok@gd\endcsname{\def\PY@tc##1{\textcolor[rgb]{0.63,0.00,0.00}{##1}}}
\expandafter\def\csname PY@tok@gi\endcsname{\def\PY@tc##1{\textcolor[rgb]{0.00,0.63,0.00}{##1}}}
\expandafter\def\csname PY@tok@gr\endcsname{\def\PY@tc##1{\textcolor[rgb]{1.00,0.00,0.00}{##1}}}
\expandafter\def\csname PY@tok@ge\endcsname{\let\PY@it=\textit}
\expandafter\def\csname PY@tok@gs\endcsname{\let\PY@bf=\textbf}
\expandafter\def\csname PY@tok@gp\endcsname{\let\PY@bf=\textbf\def\PY@tc##1{\textcolor[rgb]{0.00,0.00,0.50}{##1}}}
\expandafter\def\csname PY@tok@go\endcsname{\def\PY@tc##1{\textcolor[rgb]{0.53,0.53,0.53}{##1}}}
\expandafter\def\csname PY@tok@gt\endcsname{\def\PY@tc##1{\textcolor[rgb]{0.00,0.27,0.87}{##1}}}
\expandafter\def\csname PY@tok@err\endcsname{\def\PY@bc##1{\setlength{\fboxsep}{0pt}\fcolorbox[rgb]{1.00,0.00,0.00}{1,1,1}{\strut ##1}}}
\expandafter\def\csname PY@tok@kc\endcsname{\let\PY@bf=\textbf\def\PY@tc##1{\textcolor[rgb]{0.00,0.50,0.00}{##1}}}
\expandafter\def\csname PY@tok@kd\endcsname{\let\PY@bf=\textbf\def\PY@tc##1{\textcolor[rgb]{0.00,0.50,0.00}{##1}}}
\expandafter\def\csname PY@tok@kn\endcsname{\let\PY@bf=\textbf\def\PY@tc##1{\textcolor[rgb]{0.00,0.50,0.00}{##1}}}
\expandafter\def\csname PY@tok@kr\endcsname{\let\PY@bf=\textbf\def\PY@tc##1{\textcolor[rgb]{0.00,0.50,0.00}{##1}}}
\expandafter\def\csname PY@tok@bp\endcsname{\def\PY@tc##1{\textcolor[rgb]{0.00,0.50,0.00}{##1}}}
\expandafter\def\csname PY@tok@fm\endcsname{\def\PY@tc##1{\textcolor[rgb]{0.00,0.00,1.00}{##1}}}
\expandafter\def\csname PY@tok@vc\endcsname{\def\PY@tc##1{\textcolor[rgb]{0.10,0.09,0.49}{##1}}}
\expandafter\def\csname PY@tok@vg\endcsname{\def\PY@tc##1{\textcolor[rgb]{0.10,0.09,0.49}{##1}}}
\expandafter\def\csname PY@tok@vi\endcsname{\def\PY@tc##1{\textcolor[rgb]{0.10,0.09,0.49}{##1}}}
\expandafter\def\csname PY@tok@vm\endcsname{\def\PY@tc##1{\textcolor[rgb]{0.10,0.09,0.49}{##1}}}
\expandafter\def\csname PY@tok@sa\endcsname{\def\PY@tc##1{\textcolor[rgb]{0.73,0.13,0.13}{##1}}}
\expandafter\def\csname PY@tok@sb\endcsname{\def\PY@tc##1{\textcolor[rgb]{0.73,0.13,0.13}{##1}}}
\expandafter\def\csname PY@tok@sc\endcsname{\def\PY@tc##1{\textcolor[rgb]{0.73,0.13,0.13}{##1}}}
\expandafter\def\csname PY@tok@dl\endcsname{\def\PY@tc##1{\textcolor[rgb]{0.73,0.13,0.13}{##1}}}
\expandafter\def\csname PY@tok@s2\endcsname{\def\PY@tc##1{\textcolor[rgb]{0.73,0.13,0.13}{##1}}}
\expandafter\def\csname PY@tok@sh\endcsname{\def\PY@tc##1{\textcolor[rgb]{0.73,0.13,0.13}{##1}}}
\expandafter\def\csname PY@tok@s1\endcsname{\def\PY@tc##1{\textcolor[rgb]{0.73,0.13,0.13}{##1}}}
\expandafter\def\csname PY@tok@mb\endcsname{\def\PY@tc##1{\textcolor[rgb]{0.40,0.40,0.40}{##1}}}
\expandafter\def\csname PY@tok@mf\endcsname{\def\PY@tc##1{\textcolor[rgb]{0.40,0.40,0.40}{##1}}}
\expandafter\def\csname PY@tok@mh\endcsname{\def\PY@tc##1{\textcolor[rgb]{0.40,0.40,0.40}{##1}}}
\expandafter\def\csname PY@tok@mi\endcsname{\def\PY@tc##1{\textcolor[rgb]{0.40,0.40,0.40}{##1}}}
\expandafter\def\csname PY@tok@il\endcsname{\def\PY@tc##1{\textcolor[rgb]{0.40,0.40,0.40}{##1}}}
\expandafter\def\csname PY@tok@mo\endcsname{\def\PY@tc##1{\textcolor[rgb]{0.40,0.40,0.40}{##1}}}
\expandafter\def\csname PY@tok@ch\endcsname{\let\PY@it=\textit\def\PY@tc##1{\textcolor[rgb]{0.25,0.50,0.50}{##1}}}
\expandafter\def\csname PY@tok@cm\endcsname{\let\PY@it=\textit\def\PY@tc##1{\textcolor[rgb]{0.25,0.50,0.50}{##1}}}
\expandafter\def\csname PY@tok@cpf\endcsname{\let\PY@it=\textit\def\PY@tc##1{\textcolor[rgb]{0.25,0.50,0.50}{##1}}}
\expandafter\def\csname PY@tok@c1\endcsname{\let\PY@it=\textit\def\PY@tc##1{\textcolor[rgb]{0.25,0.50,0.50}{##1}}}
\expandafter\def\csname PY@tok@cs\endcsname{\let\PY@it=\textit\def\PY@tc##1{\textcolor[rgb]{0.25,0.50,0.50}{##1}}}

\def\PYZbs{\char`\\}
\def\PYZus{\char`\_}
\def\PYZob{\char`\{}
\def\PYZcb{\char`\}}
\def\PYZca{\char`\^}
\def\PYZam{\char`\&}
\def\PYZlt{\char`\<}
\def\PYZgt{\char`\>}
\def\PYZsh{\char`\#}
\def\PYZpc{\char`\%}
\def\PYZdl{\char`\$}
\def\PYZhy{\char`\-}
\def\PYZsq{\char`\'}
\def\PYZdq{\char`\"}
\def\PYZti{\char`\~}
% for compatibility with earlier versions
\def\PYZat{@}
\def\PYZlb{[}
\def\PYZrb{]}
\makeatother


    % Exact colors from NB
    \definecolor{incolor}{rgb}{0.0, 0.0, 0.5}
    \definecolor{outcolor}{rgb}{0.545, 0.0, 0.0}



    
    % Prevent overflowing lines due to hard-to-break entities
    \sloppy 
    % Setup hyperref package
    \hypersetup{
      breaklinks=true,  % so long urls are correctly broken across lines
      colorlinks=true,
      urlcolor=urlcolor,
      linkcolor=linkcolor,
      citecolor=citecolor,
      }
    % Slightly bigger margins than the latex defaults
    
    \geometry{verbose,tmargin=1in,bmargin=1in,lmargin=1in,rmargin=1in}
    
    

    \begin{document}
    
    
    \maketitle
    
    

    
    \section{4 Pre-Processing and Training
Data}\label{pre-processing-and-training-data}

    \subsection{4.1 Contents}\label{contents}

\begin{itemize}
\tightlist
\item
  Section \ref{4_pre-processing_and_training_data}
\item
  Section \ref{41_contents}
\item
  Section \ref{42_introduction}
\item
  Section \ref{43_imports}
\item
  Section \ref{44_load_data}
\item
  Section \ref{45_extract_big_mountain_data}
\item
  Section \ref{46_traintest_split}
\item
  Section \ref{47_initial_not-even-a-model}

  \begin{itemize}
  \tightlist
  \item
    Section \ref{471_metrics}
  \item
    Section \ref{4711_r-squared_or_coefficient_of_determination}
  \item
    Section \ref{4712_mean_absolute_error}
  \item
    Section \ref{4713_mean_squared_error}
  \item
    Section \ref{472_sklearn_metrics}

    \begin{itemize}
    \tightlist
    \item
      Section \ref{47201_r-squared}
    \item
      Section \ref{47202_mean_absolute_error}
    \item
      Section \ref{47203_mean_squared_error}
    \end{itemize}
  \item
    Section \ref{473_note_on_calculating_metrics}
  \end{itemize}
\item
  Section \ref{48_initial_models}

  \begin{itemize}
  \tightlist
  \item
    Section \ref{481_imputing_missing_feature_predictor_values}
  \item
    Section \ref{4811_impute_missing_values_with_median}

    \begin{itemize}
    \tightlist
    \item
      Section \ref{48111_learn_the_values_to_impute_from_the_train_set}
    \item
      Section \ref{48112_apply_the_imputation_to_both_train_and_test_splits}
    \item
      Section \ref{48113_scale_the_data}
    \item
      Section \ref{48114_train_the_model_on_the_train_split}
    \item
      Section \ref{48115_make_predictions_using_the_model_on_both_train_and_test_splits}
    \item
      Section \ref{48116_assess_model_performance}
    \end{itemize}
  \item
    Section \ref{4812_impute_missing_values_with_the_mean}

    \begin{itemize}
    \tightlist
    \item
      Section \ref{48121_learn_the_values_to_impute_from_the_train_set}
    \item
      Section \ref{48122_apply_the_imputation_to_both_train_and_test_splits}
    \item
      Section \ref{48123_scale_the_data}
    \item
      Section \ref{48124_train_the_model_on_the_train_split}
    \item
      Section \ref{48125_make_predictions_using_the_model_on_both_train_and_test_splits}
    \item
      Section \ref{48126_assess_model_performance}
    \end{itemize}
  \item
    Section \ref{482_pipelines}
  \item
    Section \ref{4821_define_the_pipeline}
  \item
    Section \ref{4822_fit_the_pipeline}
  \item
    Section \ref{4823_make_predictions_on_the_train_and_test_sets}
  \item
    Section \ref{4824_assess_performance}
  \end{itemize}
\item
  Section \ref{49_refining_the_linear_model}

  \begin{itemize}
  \tightlist
  \item
    Section \ref{491_define_the_pipeline}
  \item
    Section \ref{492_fit_the_pipeline}
  \item
    Section \ref{493_assess_performance_on_the_train_and_test_set}
  \item
    Section \ref{494_define_a_new_pipeline_to_select_a_different_number_of_features}
  \item
    Section \ref{495_fit_the_pipeline}
  \item
    Section \ref{496_assess_performance_on_train_and_test_data}
  \item
    Section \ref{497_assessing_performance_using_cross-validation}
  \item
    Section \ref{498_hyperparameter_search_using_gridsearchcv}
  \end{itemize}
\item
  Section \ref{410_random_forest_model}

  \begin{itemize}
  \tightlist
  \item
    Section \ref{4101_define_the_pipeline}
  \item
    Section \ref{4102_fit_and_assess_performance_using_cross-validation}
  \item
    Section \ref{4103_hyperparameter_search_using_gridsearchcv}
  \end{itemize}
\item
  Section \ref{411_final_model_selection}

  \begin{itemize}
  \tightlist
  \item
    Section \ref{4111_linear_regression_model_performance}
  \item
    Section \ref{4112_random_forest_regression_model_performance}
  \item
    Section \ref{4113_conclusion}
  \end{itemize}
\item
  Section \ref{412_data_quantity_assessment}
\item
  Section \ref{413_save_best_model_object_from_pipeline}
\item
  Section \ref{414_summary}
\end{itemize}

    \subsection{4.2 Introduction}\label{introduction}

    In preceding notebooks, performed preliminary assessments of data
quality and refined the question to be answered. You found a small
number of data values that gave clear choices about whether to replace
values or drop a whole row. You determined that predicting the adult
weekend ticket price was your primary aim. You threw away records with
missing price data, but not before making the most of the other
available data to look for any patterns between the states. You didn't
see any and decided to treat all states equally; the state label didn't
seem to be particularly useful.

In this notebook you'll start to build machine learning models. Before
even starting with learning a machine learning model, however, start by
considering how useful the mean value is as a predictor. This is more
than just a pedagogical device. You never want to go to stakeholders
with a machine learning model only to have the CEO point out that it
performs worse than just guessing the average! Your first model is a
baseline performance comparitor for any subsequent model. You then build
up the process of efficiently and robustly creating and assessing models
against it. The development we lay out may be little slower than in the
real world, but this step of the capstone is definitely more than just
instructional. It is good practice to build up an understanding that the
machine learning pipelines you build work as expected. You can validate
steps with your own functions for checking expected equivalence between,
say, pandas and sklearn implementations.

    \subsection{4.3 Imports}\label{imports}

    \begin{Verbatim}[commandchars=\\\{\}]
{\color{incolor}In [{\color{incolor}1}]:} \PY{k+kn}{import} \PY{n+nn}{pandas} \PY{k}{as} \PY{n+nn}{pd}
        \PY{k+kn}{import} \PY{n+nn}{numpy} \PY{k}{as} \PY{n+nn}{np}
        \PY{k+kn}{import} \PY{n+nn}{os}
        \PY{k+kn}{import} \PY{n+nn}{pickle}
        \PY{k+kn}{import} \PY{n+nn}{matplotlib}\PY{n+nn}{.}\PY{n+nn}{pyplot} \PY{k}{as} \PY{n+nn}{plt}
        \PY{k+kn}{import} \PY{n+nn}{seaborn} \PY{k}{as} \PY{n+nn}{sns}
        \PY{k+kn}{from} \PY{n+nn}{sklearn} \PY{k}{import} \PY{n}{\PYZus{}\PYZus{}version\PYZus{}\PYZus{}} \PY{k}{as} \PY{n}{sklearn\PYZus{}version}
        \PY{k+kn}{from} \PY{n+nn}{sklearn}\PY{n+nn}{.}\PY{n+nn}{decomposition} \PY{k}{import} \PY{n}{PCA}
        \PY{k+kn}{from} \PY{n+nn}{sklearn}\PY{n+nn}{.}\PY{n+nn}{preprocessing} \PY{k}{import} \PY{n}{scale}
        \PY{k+kn}{from} \PY{n+nn}{sklearn}\PY{n+nn}{.}\PY{n+nn}{model\PYZus{}selection} \PY{k}{import} \PY{n}{train\PYZus{}test\PYZus{}split}\PY{p}{,} \PY{n}{cross\PYZus{}validate}\PY{p}{,} \PY{n}{GridSearchCV}\PY{p}{,} \PY{n}{learning\PYZus{}curve}
        \PY{k+kn}{from} \PY{n+nn}{sklearn}\PY{n+nn}{.}\PY{n+nn}{preprocessing} \PY{k}{import} \PY{n}{StandardScaler}\PY{p}{,} \PY{n}{MinMaxScaler}
        \PY{k+kn}{from} \PY{n+nn}{sklearn}\PY{n+nn}{.}\PY{n+nn}{dummy} \PY{k}{import} \PY{n}{DummyRegressor}
        \PY{k+kn}{from} \PY{n+nn}{sklearn}\PY{n+nn}{.}\PY{n+nn}{linear\PYZus{}model} \PY{k}{import} \PY{n}{LinearRegression}
        \PY{k+kn}{from} \PY{n+nn}{sklearn}\PY{n+nn}{.}\PY{n+nn}{ensemble} \PY{k}{import} \PY{n}{RandomForestRegressor}
        \PY{k+kn}{from} \PY{n+nn}{sklearn}\PY{n+nn}{.}\PY{n+nn}{metrics} \PY{k}{import} \PY{n}{r2\PYZus{}score}\PY{p}{,} \PY{n}{mean\PYZus{}squared\PYZus{}error}\PY{p}{,} \PY{n}{mean\PYZus{}absolute\PYZus{}error}
        \PY{k+kn}{from} \PY{n+nn}{sklearn}\PY{n+nn}{.}\PY{n+nn}{pipeline} \PY{k}{import} \PY{n}{make\PYZus{}pipeline}
        \PY{k+kn}{from} \PY{n+nn}{sklearn}\PY{n+nn}{.}\PY{n+nn}{impute} \PY{k}{import} \PY{n}{SimpleImputer}
        \PY{k+kn}{from} \PY{n+nn}{sklearn}\PY{n+nn}{.}\PY{n+nn}{feature\PYZus{}selection} \PY{k}{import} \PY{n}{SelectKBest}\PY{p}{,} \PY{n}{f\PYZus{}regression}
        \PY{k+kn}{import} \PY{n+nn}{datetime}
        
        \PY{k+kn}{from} \PY{n+nn}{library}\PY{n+nn}{.}\PY{n+nn}{sb\PYZus{}utils} \PY{k}{import} \PY{n}{save\PYZus{}file}
\end{Verbatim}


    \subsection{4.4 Load Data}\label{load-data}

    \begin{Verbatim}[commandchars=\\\{\}]
{\color{incolor}In [{\color{incolor}2}]:} \PY{n}{ski\PYZus{}data} \PY{o}{=} \PY{n}{pd}\PY{o}{.}\PY{n}{read\PYZus{}csv}\PY{p}{(}\PY{l+s+s1}{\PYZsq{}}\PY{l+s+s1}{../data/ski\PYZus{}data\PYZus{}step3\PYZus{}features.csv}\PY{l+s+s1}{\PYZsq{}}\PY{p}{)}
        \PY{n}{ski\PYZus{}data}\PY{o}{.}\PY{n}{head}\PY{p}{(}\PY{p}{)}\PY{o}{.}\PY{n}{T}
\end{Verbatim}


\begin{Verbatim}[commandchars=\\\{\}]
{\color{outcolor}Out[{\color{outcolor}2}]:}                                                  0                    1  \textbackslash{}
        Name                                Alyeska Resort  Eaglecrest Ski Area   
        Region                                      Alaska               Alaska   
        state                                       Alaska               Alaska   
        summit\_elev                                   3939                 2600   
        vertical\_drop                                 2500                 1540   
        base\_elev                                      250                 1200   
        trams                                            1                    0   
        fastSixes                                        0                    0   
        fastQuads                                        2                    0   
        quad                                             2                    0   
        triple                                           0                    0   
        double                                           0                    4   
        surface                                          2                    0   
        total\_chairs                                     7                    4   
        Runs                                            76                   36   
        TerrainParks                                     2                    1   
        LongestRun\_mi                                    1                    2   
        SkiableTerrain\_ac                             1610                  640   
        Snow Making\_ac                                 113                   60   
        daysOpenLastYear                               150                   45   
        yearsOpen                                       60                   44   
        averageSnowfall                                669                  350   
        AdultWeekend                                    85                   53   
        projectedDaysOpen                              150                   90   
        NightSkiing\_ac                                 550                  NaN   
        resorts\_per\_state                                3                    3   
        resorts\_per\_100kcapita                    0.410091             0.410091   
        resorts\_per\_100ksq\_mile                   0.450867             0.450867   
        resort\_skiable\_area\_ac\_state\_ratio         0.70614             0.280702   
        resort\_days\_open\_state\_ratio              0.434783             0.130435   
        resort\_terrain\_park\_state\_ratio                0.5                 0.25   
        resort\_night\_skiing\_state\_ratio           0.948276                  NaN   
        total\_chairs\_runs\_ratio                  0.0921053             0.111111   
        total\_chairs\_skiable\_ratio              0.00434783              0.00625   
        fastQuads\_runs\_ratio                     0.0263158                    0   
        fastQuads\_skiable\_ratio                 0.00124224                    0   
        
                                                           2                 3  \textbackslash{}
        Name                                Hilltop Ski Area  Arizona Snowbowl   
        Region                                        Alaska           Arizona   
        state                                         Alaska           Arizona   
        summit\_elev                                     2090             11500   
        vertical\_drop                                    294              2300   
        base\_elev                                       1796              9200   
        trams                                              0                 0   
        fastSixes                                          0                 1   
        fastQuads                                          0                 0   
        quad                                               0                 2   
        triple                                             1                 2   
        double                                             0                 1   
        surface                                            2                 2   
        total\_chairs                                       3                 8   
        Runs                                              13                55   
        TerrainParks                                       1                 4   
        LongestRun\_mi                                      1                 2   
        SkiableTerrain\_ac                                 30               777   
        Snow Making\_ac                                    30               104   
        daysOpenLastYear                                 150               122   
        yearsOpen                                         36                81   
        averageSnowfall                                   69               260   
        AdultWeekend                                      34                89   
        projectedDaysOpen                                152               122   
        NightSkiing\_ac                                    30               NaN   
        resorts\_per\_state                                  3                 2   
        resorts\_per\_100kcapita                      0.410091         0.0274774   
        resorts\_per\_100ksq\_mile                     0.450867           1.75454   
        resort\_skiable\_area\_ac\_state\_ratio         0.0131579          0.492708   
        resort\_days\_open\_state\_ratio                0.434783          0.514768   
        resort\_terrain\_park\_state\_ratio                 0.25          0.666667   
        resort\_night\_skiing\_state\_ratio            0.0517241               NaN   
        total\_chairs\_runs\_ratio                     0.230769          0.145455   
        total\_chairs\_skiable\_ratio                       0.1          0.010296   
        fastQuads\_runs\_ratio                               0                 0   
        fastQuads\_skiable\_ratio                            0                 0   
        
                                                              4  
        Name                                Sunrise Park Resort  
        Region                                          Arizona  
        state                                           Arizona  
        summit\_elev                                       11100  
        vertical\_drop                                      1800  
        base\_elev                                          9200  
        trams                                                 0  
        fastSixes                                             0  
        fastQuads                                             1  
        quad                                                  2  
        triple                                                3  
        double                                                1  
        surface                                               0  
        total\_chairs                                          7  
        Runs                                                 65  
        TerrainParks                                          2  
        LongestRun\_mi                                       1.2  
        SkiableTerrain\_ac                                   800  
        Snow Making\_ac                                       80  
        daysOpenLastYear                                    115  
        yearsOpen                                            49  
        averageSnowfall                                     250  
        AdultWeekend                                         78  
        projectedDaysOpen                                   104  
        NightSkiing\_ac                                       80  
        resorts\_per\_state                                     2  
        resorts\_per\_100kcapita                        0.0274774  
        resorts\_per\_100ksq\_mile                         1.75454  
        resort\_skiable\_area\_ac\_state\_ratio             0.507292  
        resort\_days\_open\_state\_ratio                   0.485232  
        resort\_terrain\_park\_state\_ratio                0.333333  
        resort\_night\_skiing\_state\_ratio                       1  
        total\_chairs\_runs\_ratio                        0.107692  
        total\_chairs\_skiable\_ratio                      0.00875  
        fastQuads\_runs\_ratio                          0.0153846  
        fastQuads\_skiable\_ratio                         0.00125  
\end{Verbatim}
            
    \subsection{4.5 Extract Big Mountain
Data}\label{extract-big-mountain-data}

    Big Mountain is your resort. Separate it from the rest of the data to
use later.

    \begin{Verbatim}[commandchars=\\\{\}]
{\color{incolor}In [{\color{incolor}3}]:} \PY{n}{big\PYZus{}mountain} \PY{o}{=} \PY{n}{ski\PYZus{}data}\PY{p}{[}\PY{n}{ski\PYZus{}data}\PY{o}{.}\PY{n}{Name} \PY{o}{==} \PY{l+s+s1}{\PYZsq{}}\PY{l+s+s1}{Big Mountain Resort}\PY{l+s+s1}{\PYZsq{}}\PY{p}{]}
\end{Verbatim}


    \begin{Verbatim}[commandchars=\\\{\}]
{\color{incolor}In [{\color{incolor}4}]:} \PY{n}{big\PYZus{}mountain}\PY{o}{.}\PY{n}{T}
\end{Verbatim}


\begin{Verbatim}[commandchars=\\\{\}]
{\color{outcolor}Out[{\color{outcolor}4}]:}                                                     124
        Name                                Big Mountain Resort
        Region                                          Montana
        state                                           Montana
        summit\_elev                                        6817
        vertical\_drop                                      2353
        base\_elev                                          4464
        trams                                                 0
        fastSixes                                             0
        fastQuads                                             3
        quad                                                  2
        triple                                                6
        double                                                0
        surface                                               3
        total\_chairs                                         14
        Runs                                                105
        TerrainParks                                          4
        LongestRun\_mi                                       3.3
        SkiableTerrain\_ac                                  3000
        Snow Making\_ac                                      600
        daysOpenLastYear                                    123
        yearsOpen                                            72
        averageSnowfall                                     333
        AdultWeekend                                         81
        projectedDaysOpen                                   123
        NightSkiing\_ac                                      600
        resorts\_per\_state                                    11
        resorts\_per\_100kcapita                          1.02921
        resorts\_per\_100ksq\_mile                         7.48096
        resort\_skiable\_area\_ac\_state\_ratio             0.192184
        resort\_days\_open\_state\_ratio                   0.152416
        resort\_terrain\_park\_state\_ratio                0.210526
        resort\_night\_skiing\_state\_ratio                 0.84507
        total\_chairs\_runs\_ratio                        0.133333
        total\_chairs\_skiable\_ratio                   0.00466667
        fastQuads\_runs\_ratio                          0.0285714
        fastQuads\_skiable\_ratio                           0.001
\end{Verbatim}
            
    \begin{Verbatim}[commandchars=\\\{\}]
{\color{incolor}In [{\color{incolor}5}]:} \PY{n}{ski\PYZus{}data}\PY{o}{.}\PY{n}{shape}
\end{Verbatim}


\begin{Verbatim}[commandchars=\\\{\}]
{\color{outcolor}Out[{\color{outcolor}5}]:} (277, 36)
\end{Verbatim}
            
    \begin{Verbatim}[commandchars=\\\{\}]
{\color{incolor}In [{\color{incolor}6}]:} \PY{n}{ski\PYZus{}data} \PY{o}{=} \PY{n}{ski\PYZus{}data}\PY{p}{[}\PY{n}{ski\PYZus{}data}\PY{o}{.}\PY{n}{Name} \PY{o}{!=} \PY{l+s+s1}{\PYZsq{}}\PY{l+s+s1}{Big Mountain Resort}\PY{l+s+s1}{\PYZsq{}}\PY{p}{]}
\end{Verbatim}


    \begin{Verbatim}[commandchars=\\\{\}]
{\color{incolor}In [{\color{incolor}7}]:} \PY{n}{ski\PYZus{}data}\PY{o}{.}\PY{n}{shape}
\end{Verbatim}


\begin{Verbatim}[commandchars=\\\{\}]
{\color{outcolor}Out[{\color{outcolor}7}]:} (276, 36)
\end{Verbatim}
            
    \subsection{4.6 Train/Test Split}\label{traintest-split}

    So far, you've treated ski resort data as a single entity. In machine
learning, when you train your model on all of your data, you end up with
no data set aside to evaluate model performance. You could keep making
more and more complex models that fit the data better and better and not
realise you were overfitting to that one set of samples. By partitioning
the data into training and testing splits, without letting a model (or
missing-value imputation) learn anything about the test split, you have
a somewhat independent assessment of how your model might perform in the
future. An often overlooked subtlety here is that people all too
frequently use the test set to assess model performance \emph{and then
compare multiple models to pick the best}. This means their overall
model selection process is fitting to one specific data set, now the
test split. You could keep going, trying to get better and better
performance on that one data set, but that's where cross-validation
becomes especially useful. While training models, a test split is very
useful as a final check on expected future performance.

    What partition sizes would you have with a 70/30 train/test split?

    \begin{Verbatim}[commandchars=\\\{\}]
{\color{incolor}In [{\color{incolor}8}]:} \PY{n+nb}{len}\PY{p}{(}\PY{n}{ski\PYZus{}data}\PY{p}{)} \PY{o}{*} \PY{o}{.}\PY{l+m+mi}{7}\PY{p}{,} \PY{n+nb}{len}\PY{p}{(}\PY{n}{ski\PYZus{}data}\PY{p}{)} \PY{o}{*} \PY{o}{.}\PY{l+m+mi}{3}
\end{Verbatim}


\begin{Verbatim}[commandchars=\\\{\}]
{\color{outcolor}Out[{\color{outcolor}8}]:} (193.2, 82.8)
\end{Verbatim}
            
    \begin{Verbatim}[commandchars=\\\{\}]
{\color{incolor}In [{\color{incolor}9}]:} \PY{n}{X\PYZus{}train}\PY{p}{,} \PY{n}{X\PYZus{}test}\PY{p}{,} \PY{n}{y\PYZus{}train}\PY{p}{,} \PY{n}{y\PYZus{}test} \PY{o}{=} \PY{n}{train\PYZus{}test\PYZus{}split}\PY{p}{(}\PY{n}{ski\PYZus{}data}\PY{o}{.}\PY{n}{drop}\PY{p}{(}\PY{n}{columns}\PY{o}{=}\PY{l+s+s1}{\PYZsq{}}\PY{l+s+s1}{AdultWeekend}\PY{l+s+s1}{\PYZsq{}}\PY{p}{)}\PY{p}{,} 
                                                            \PY{n}{ski\PYZus{}data}\PY{o}{.}\PY{n}{AdultWeekend}\PY{p}{,} \PY{n}{test\PYZus{}size}\PY{o}{=}\PY{l+m+mf}{0.3}\PY{p}{,} 
                                                            \PY{n}{random\PYZus{}state}\PY{o}{=}\PY{l+m+mi}{47}\PY{p}{)}
\end{Verbatim}


    \begin{Verbatim}[commandchars=\\\{\}]
{\color{incolor}In [{\color{incolor}10}]:} \PY{n}{X\PYZus{}train}\PY{o}{.}\PY{n}{shape}\PY{p}{,} \PY{n}{X\PYZus{}test}\PY{o}{.}\PY{n}{shape}
\end{Verbatim}


\begin{Verbatim}[commandchars=\\\{\}]
{\color{outcolor}Out[{\color{outcolor}10}]:} ((193, 35), (83, 35))
\end{Verbatim}
            
    \begin{Verbatim}[commandchars=\\\{\}]
{\color{incolor}In [{\color{incolor}11}]:} \PY{n}{X\PYZus{}train}\PY{o}{.}\PY{n}{head}\PY{p}{(}\PY{p}{)}
\end{Verbatim}


\begin{Verbatim}[commandchars=\\\{\}]
{\color{outcolor}Out[{\color{outcolor}11}]:}                          Name          Region           state  summit\_elev  \textbackslash{}
         108     Powder Ridge Ski Area       Minnesota       Minnesota          790   
         96              The Homestead        Michigan        Michigan          900   
         189     Beech Mountain Resort  North Carolina  North Carolina         5506   
         232  Solitude Mountain Resort  Salt Lake City            Utah        10488   
         1         Eaglecrest Ski Area          Alaska          Alaska         2600   
         
              vertical\_drop  base\_elev  trams  fastSixes  fastQuads  quad  {\ldots}  \textbackslash{}
         108            300        500      0          0          0     1  {\ldots}   
         96             320        580      0          0          0     0  {\ldots}   
         189            830       4675      0          0          0     3  {\ldots}   
         232           2494       7994      0          0          4     2  {\ldots}   
         1             1540       1200      0          0          0     0  {\ldots}   
         
              resorts\_per\_100kcapita  resorts\_per\_100ksq\_mile  \textbackslash{}
         108                0.248243                16.103800   
         96                 0.250329                25.849412   
         189                0.057208                11.148479   
         232                0.374303                14.134775   
         1                  0.410091                 0.450867   
         
              resort\_skiable\_area\_ac\_state\_ratio  resort\_days\_open\_state\_ratio  \textbackslash{}
         108                            0.038462                      0.065101   
         96                             0.004276                      0.022760   
         189                            0.256757                      0.193676   
         232                            0.051706                      0.114836   
         1                              0.280702                      0.130435   
         
              resort\_terrain\_park\_state\_ratio  resort\_night\_skiing\_state\_ratio  \textbackslash{}
         108                         0.137931                         0.058824   
         96                          0.020000                         0.010146   
         189                         0.111111                         0.283582   
         232                              NaN                              NaN   
         1                           0.250000                              NaN   
         
              total\_chairs\_runs\_ratio  total\_chairs\_skiable\_ratio  \textbackslash{}
         108                 0.400000                    0.100000   
         96                  0.333333                    0.312500   
         189                 0.470588                    0.084211   
         232                 0.112500                    0.007500   
         1                   0.111111                    0.006250   
         
              fastQuads\_runs\_ratio  fastQuads\_skiable\_ratio  
         108                  0.00                 0.000000  
         96                   0.00                 0.000000  
         189                  0.00                 0.000000  
         232                  0.05                 0.003333  
         1                    0.00                 0.000000  
         
         [5 rows x 35 columns]
\end{Verbatim}
            
    \begin{Verbatim}[commandchars=\\\{\}]
{\color{incolor}In [{\color{incolor}12}]:} \PY{n}{y\PYZus{}train}\PY{o}{.}\PY{n}{shape}\PY{p}{,} \PY{n}{y\PYZus{}test}\PY{o}{.}\PY{n}{shape}
\end{Verbatim}


\begin{Verbatim}[commandchars=\\\{\}]
{\color{outcolor}Out[{\color{outcolor}12}]:} ((193,), (83,))
\end{Verbatim}
            
    \begin{Verbatim}[commandchars=\\\{\}]
{\color{incolor}In [{\color{incolor}13}]:} \PY{n}{y\PYZus{}train}
\end{Verbatim}


\begin{Verbatim}[commandchars=\\\{\}]
{\color{outcolor}Out[{\color{outcolor}13}]:} 108     48.0
         96      50.0
         189     68.0
         232    119.0
         1       53.0
                {\ldots}  
         23      89.0
         180     55.0
         72      71.0
         265     47.0
         136     58.0
         Name: AdultWeekend, Length: 193, dtype: float64
\end{Verbatim}
            
    \begin{Verbatim}[commandchars=\\\{\}]
{\color{incolor}In [{\color{incolor}14}]:} \PY{c+c1}{\PYZsh{}Code task 1\PYZsh{}}
         \PY{c+c1}{\PYZsh{}Save the \PYZsq{}Name\PYZsq{}, \PYZsq{}state\PYZsq{}, and \PYZsq{}Region\PYZsq{} columns from the train/test data into names\PYZus{}train and names\PYZus{}test}
         \PY{c+c1}{\PYZsh{}Then drop those columns from `X\PYZus{}train` and `X\PYZus{}test`. Use \PYZsq{}inplace=True\PYZsq{}}
         \PY{n}{names\PYZus{}list} \PY{o}{=} \PY{p}{[}\PY{l+s+s1}{\PYZsq{}}\PY{l+s+s1}{Name}\PY{l+s+s1}{\PYZsq{}}\PY{p}{,} \PY{l+s+s1}{\PYZsq{}}\PY{l+s+s1}{state}\PY{l+s+s1}{\PYZsq{}}\PY{p}{,} \PY{l+s+s1}{\PYZsq{}}\PY{l+s+s1}{Region}\PY{l+s+s1}{\PYZsq{}}\PY{p}{]}
         \PY{n}{names\PYZus{}train} \PY{o}{=} \PY{n}{X\PYZus{}train}\PY{p}{[}\PY{n}{names\PYZus{}list}\PY{p}{]}
         \PY{n}{names\PYZus{}test} \PY{o}{=} \PY{n}{X\PYZus{}test}\PY{p}{[}\PY{n}{names\PYZus{}list}\PY{p}{]}
         \PY{n}{X\PYZus{}train}\PY{o}{.}\PY{n}{drop}\PY{p}{(}\PY{n}{columns}\PY{o}{=}\PY{n}{names\PYZus{}list}\PY{p}{,} \PY{n}{inplace}\PY{o}{=}\PY{k+kc}{True}\PY{p}{)}
         \PY{n}{X\PYZus{}test}\PY{o}{.}\PY{n}{drop}\PY{p}{(}\PY{n}{columns}\PY{o}{=}\PY{n}{names\PYZus{}list}\PY{p}{,} \PY{n}{inplace}\PY{o}{=}\PY{k+kc}{True}\PY{p}{)}
         \PY{n}{X\PYZus{}train}\PY{o}{.}\PY{n}{shape}\PY{p}{,} \PY{n}{X\PYZus{}test}\PY{o}{.}\PY{n}{shape}
\end{Verbatim}


\begin{Verbatim}[commandchars=\\\{\}]
{\color{outcolor}Out[{\color{outcolor}14}]:} ((193, 32), (83, 32))
\end{Verbatim}
            
    \begin{Verbatim}[commandchars=\\\{\}]
{\color{incolor}In [{\color{incolor}15}]:} \PY{n}{X\PYZus{}train}\PY{o}{.}\PY{n}{head}\PY{p}{(}\PY{p}{)}
\end{Verbatim}


\begin{Verbatim}[commandchars=\\\{\}]
{\color{outcolor}Out[{\color{outcolor}15}]:}      summit\_elev  vertical\_drop  base\_elev  trams  fastSixes  fastQuads  quad  \textbackslash{}
         108          790            300        500      0          0          0     1   
         96           900            320        580      0          0          0     0   
         189         5506            830       4675      0          0          0     3   
         232        10488           2494       7994      0          0          4     2   
         1           2600           1540       1200      0          0          0     0   
         
              triple  double  surface  {\ldots}  resorts\_per\_100kcapita  \textbackslash{}
         108       0       2        3  {\ldots}                0.248243   
         96        2       1        2  {\ldots}                0.250329   
         189       0       3        2  {\ldots}                0.057208   
         232       1       1        1  {\ldots}                0.374303   
         1         0       4        0  {\ldots}                0.410091   
         
              resorts\_per\_100ksq\_mile  resort\_skiable\_area\_ac\_state\_ratio  \textbackslash{}
         108                16.103800                            0.038462   
         96                 25.849412                            0.004276   
         189                11.148479                            0.256757   
         232                14.134775                            0.051706   
         1                   0.450867                            0.280702   
         
              resort\_days\_open\_state\_ratio  resort\_terrain\_park\_state\_ratio  \textbackslash{}
         108                      0.065101                         0.137931   
         96                       0.022760                         0.020000   
         189                      0.193676                         0.111111   
         232                      0.114836                              NaN   
         1                        0.130435                         0.250000   
         
              resort\_night\_skiing\_state\_ratio  total\_chairs\_runs\_ratio  \textbackslash{}
         108                         0.058824                 0.400000   
         96                          0.010146                 0.333333   
         189                         0.283582                 0.470588   
         232                              NaN                 0.112500   
         1                                NaN                 0.111111   
         
              total\_chairs\_skiable\_ratio  fastQuads\_runs\_ratio  fastQuads\_skiable\_ratio  
         108                    0.100000                  0.00                 0.000000  
         96                     0.312500                  0.00                 0.000000  
         189                    0.084211                  0.00                 0.000000  
         232                    0.007500                  0.05                 0.003333  
         1                      0.006250                  0.00                 0.000000  
         
         [5 rows x 32 columns]
\end{Verbatim}
            
    \begin{Verbatim}[commandchars=\\\{\}]
{\color{incolor}In [{\color{incolor}16}]:} \PY{c+c1}{\PYZsh{}Code task 2\PYZsh{}}
         \PY{c+c1}{\PYZsh{}Check the `dtypes` attribute of `X\PYZus{}train` to verify all features are numeric}
         \PY{n}{X\PYZus{}train}\PY{o}{.}\PY{n}{dtypes}
\end{Verbatim}


\begin{Verbatim}[commandchars=\\\{\}]
{\color{outcolor}Out[{\color{outcolor}16}]:} summit\_elev                             int64
         vertical\_drop                           int64
         base\_elev                               int64
         trams                                   int64
         fastSixes                               int64
         fastQuads                               int64
         quad                                    int64
         triple                                  int64
         double                                  int64
         surface                                 int64
         total\_chairs                            int64
         Runs                                  float64
         TerrainParks                          float64
         LongestRun\_mi                         float64
         SkiableTerrain\_ac                     float64
         Snow Making\_ac                        float64
         daysOpenLastYear                      float64
         yearsOpen                             float64
         averageSnowfall                       float64
         projectedDaysOpen                     float64
         NightSkiing\_ac                        float64
         resorts\_per\_state                       int64
         resorts\_per\_100kcapita                float64
         resorts\_per\_100ksq\_mile               float64
         resort\_skiable\_area\_ac\_state\_ratio    float64
         resort\_days\_open\_state\_ratio          float64
         resort\_terrain\_park\_state\_ratio       float64
         resort\_night\_skiing\_state\_ratio       float64
         total\_chairs\_runs\_ratio               float64
         total\_chairs\_skiable\_ratio            float64
         fastQuads\_runs\_ratio                  float64
         fastQuads\_skiable\_ratio               float64
         dtype: object
\end{Verbatim}
            
    \begin{Verbatim}[commandchars=\\\{\}]
{\color{incolor}In [{\color{incolor}17}]:} \PY{c+c1}{\PYZsh{}Code task 3\PYZsh{}}
         \PY{c+c1}{\PYZsh{}Repeat this check for the test split in `X\PYZus{}test`}
         \PY{n}{X\PYZus{}test}\PY{o}{.}\PY{n}{dtypes}
\end{Verbatim}


\begin{Verbatim}[commandchars=\\\{\}]
{\color{outcolor}Out[{\color{outcolor}17}]:} summit\_elev                             int64
         vertical\_drop                           int64
         base\_elev                               int64
         trams                                   int64
         fastSixes                               int64
         fastQuads                               int64
         quad                                    int64
         triple                                  int64
         double                                  int64
         surface                                 int64
         total\_chairs                            int64
         Runs                                  float64
         TerrainParks                          float64
         LongestRun\_mi                         float64
         SkiableTerrain\_ac                     float64
         Snow Making\_ac                        float64
         daysOpenLastYear                      float64
         yearsOpen                             float64
         averageSnowfall                       float64
         projectedDaysOpen                     float64
         NightSkiing\_ac                        float64
         resorts\_per\_state                       int64
         resorts\_per\_100kcapita                float64
         resorts\_per\_100ksq\_mile               float64
         resort\_skiable\_area\_ac\_state\_ratio    float64
         resort\_days\_open\_state\_ratio          float64
         resort\_terrain\_park\_state\_ratio       float64
         resort\_night\_skiing\_state\_ratio       float64
         total\_chairs\_runs\_ratio               float64
         total\_chairs\_skiable\_ratio            float64
         fastQuads\_runs\_ratio                  float64
         fastQuads\_skiable\_ratio               float64
         dtype: object
\end{Verbatim}
            
    You have only numeric features in your X now!

    \subsection{4.7 Initial
Not-Even-A-Model}\label{initial-not-even-a-model}

    A good place to start is to see how good the mean is as a predictor. In
other words, what if you simply say your best guess is the average
price?

    \begin{Verbatim}[commandchars=\\\{\}]
{\color{incolor}In [{\color{incolor}18}]:} \PY{c+c1}{\PYZsh{}Code task 4\PYZsh{}}
         \PY{c+c1}{\PYZsh{}Calculate the mean of `y\PYZus{}train`}
         \PY{n}{train\PYZus{}mean} \PY{o}{=} \PY{n}{y\PYZus{}train}\PY{o}{.}\PY{n}{mean}\PY{p}{(}\PY{p}{)}
         \PY{n}{train\PYZus{}mean}
\end{Verbatim}


\begin{Verbatim}[commandchars=\\\{\}]
{\color{outcolor}Out[{\color{outcolor}18}]:} 63.811088082901556
\end{Verbatim}
            
    \texttt{sklearn}'s \texttt{DummyRegressor} easily does this:

    \begin{Verbatim}[commandchars=\\\{\}]
{\color{incolor}In [{\color{incolor}19}]:} \PY{c+c1}{\PYZsh{}Code task 5\PYZsh{}}
         \PY{c+c1}{\PYZsh{}Fit the dummy regressor on the training data}
         \PY{c+c1}{\PYZsh{}Hint, call its `.fit()` method with `X\PYZus{}train` and `y\PYZus{}train` as arguments}
         \PY{c+c1}{\PYZsh{}Then print the object\PYZsq{}s `constant\PYZus{}` attribute and verify it\PYZsq{}s the same as the mean above}
         \PY{n}{dumb\PYZus{}reg} \PY{o}{=} \PY{n}{DummyRegressor}\PY{p}{(}\PY{n}{strategy}\PY{o}{=}\PY{l+s+s1}{\PYZsq{}}\PY{l+s+s1}{mean}\PY{l+s+s1}{\PYZsq{}}\PY{p}{)}
         \PY{n}{dumb\PYZus{}reg}\PY{o}{.}\PY{n}{fit}\PY{p}{(}\PY{n}{X\PYZus{}train}\PY{p}{,} \PY{n}{y\PYZus{}train}\PY{p}{)}
         \PY{n}{dumb\PYZus{}reg}\PY{o}{.}\PY{n}{constant\PYZus{}}
\end{Verbatim}


\begin{Verbatim}[commandchars=\\\{\}]
{\color{outcolor}Out[{\color{outcolor}19}]:} array([[63.81108808]])
\end{Verbatim}
            
    How good is this? How closely does this match, or explain, the actual
values? There are many ways of assessing how good one set of values
agrees with another, which brings us to the subject of metrics.

    \subsubsection{4.7.1 Metrics}\label{metrics}

    \paragraph{4.7.1.1 R-squared, or coefficient of
determination}\label{r-squared-or-coefficient-of-determination}

    One measure is \(R^2\), the
\href{https://en.wikipedia.org/wiki/Coefficient_of_determination}{coefficient
of determination}. This is a measure of the proportion of variance in
the dependent variable (our ticket price) that is predicted by our
"model". The linked Wikipedia articles gives a nice explanation of how
negative values can arise. This is frequently a cause of confusion for
newcomers who, reasonably, ask how can a squared value be negative?

Recall the mean can be denoted by \(\bar{y}\), where

\[\bar{y} = \frac{1}{n}\sum_{i=1}^ny_i\]

and where \(y_i\) are the individual values of the dependent variable.

The total sum of squares (error), can be expressed as

\[SS_{tot} = \sum_i(y_i-\bar{y})^2\]

The above formula should be familiar as it's simply the variance without
the denominator to scale (divide) by the sample size.

The residual sum of squares is similarly defined to be

\[SS_{res} = \sum_i(y_i-\hat{y})^2\]

where \(\hat{y}\) are our predicted values for the depended variable.

The coefficient of determination, \(R^2\), here is given by

\[R^2 = 1 - \frac{SS_{res}}{SS_{tot}}\]

Putting it into words, it's one minus the ratio of the residual variance
to the original variance. Thus, the baseline model here, which always
predicts \(\bar{y}\), should give \(R^2=0\). A model that perfectly
predicts the observed values would have no residual error and so give
\(R^2=1\). Models that do worse than predicting the mean will have
increased the sum of squares of residuals and so produce a negative
\(R^2\).

    \begin{Verbatim}[commandchars=\\\{\}]
{\color{incolor}In [{\color{incolor}20}]:} \PY{c+c1}{\PYZsh{}Code task 6\PYZsh{}}
         \PY{c+c1}{\PYZsh{}Calculate the R\PYZca{}2 as defined above}
         \PY{k}{def} \PY{n+nf}{r\PYZus{}squared}\PY{p}{(}\PY{n}{y}\PY{p}{,} \PY{n}{ypred}\PY{p}{)}\PY{p}{:}
             \PY{l+s+sd}{\PYZdq{}\PYZdq{}\PYZdq{}R\PYZhy{}squared score.}
         \PY{l+s+sd}{    }
         \PY{l+s+sd}{    Calculate the R\PYZhy{}squared, or coefficient of determination, of the input.}
         \PY{l+s+sd}{    }
         \PY{l+s+sd}{    Arguments:}
         \PY{l+s+sd}{    y \PYZhy{}\PYZhy{} the observed values}
         \PY{l+s+sd}{    ypred \PYZhy{}\PYZhy{} the predicted values}
         \PY{l+s+sd}{    \PYZdq{}\PYZdq{}\PYZdq{}}
             \PY{n}{ybar} \PY{o}{=} \PY{n}{np}\PY{o}{.}\PY{n}{sum}\PY{p}{(}\PY{n}{y}\PY{p}{)} \PY{o}{/} \PY{n+nb}{len}\PY{p}{(}\PY{n}{y}\PY{p}{)} \PY{c+c1}{\PYZsh{}yes, we could use np.mean(y)}
             \PY{n}{sum\PYZus{}sq\PYZus{}tot} \PY{o}{=} \PY{n}{np}\PY{o}{.}\PY{n}{sum}\PY{p}{(}\PY{p}{(}\PY{n}{y} \PY{o}{\PYZhy{}} \PY{n}{ybar}\PY{p}{)}\PY{o}{*}\PY{o}{*}\PY{l+m+mi}{2}\PY{p}{)} \PY{c+c1}{\PYZsh{}total sum of squares error}
             \PY{n}{sum\PYZus{}sq\PYZus{}res} \PY{o}{=} \PY{n}{np}\PY{o}{.}\PY{n}{sum}\PY{p}{(}\PY{p}{(}\PY{n}{y} \PY{o}{\PYZhy{}} \PY{n}{ypred}\PY{p}{)}\PY{o}{*}\PY{o}{*}\PY{l+m+mi}{2}\PY{p}{)} \PY{c+c1}{\PYZsh{}residual sum of squares error}
             \PY{n}{R2} \PY{o}{=} \PY{l+m+mf}{1.0} \PY{o}{\PYZhy{}} \PY{p}{(}\PY{n}{sum\PYZus{}sq\PYZus{}res} \PY{o}{/} \PY{n}{sum\PYZus{}sq\PYZus{}tot}\PY{p}{)}
             \PY{k}{return} \PY{n}{R2}
\end{Verbatim}


    Make your predictions by creating an array of length the size of the
training set with the single value of the mean.

    \begin{Verbatim}[commandchars=\\\{\}]
{\color{incolor}In [{\color{incolor}21}]:} \PY{n}{y\PYZus{}tr\PYZus{}pred\PYZus{}} \PY{o}{=} \PY{n}{train\PYZus{}mean} \PY{o}{*} \PY{n}{np}\PY{o}{.}\PY{n}{ones}\PY{p}{(}\PY{n+nb}{len}\PY{p}{(}\PY{n}{y\PYZus{}train}\PY{p}{)}\PY{p}{)}
         \PY{n}{y\PYZus{}tr\PYZus{}pred\PYZus{}}\PY{p}{[}\PY{p}{:}\PY{l+m+mi}{5}\PY{p}{]}
\end{Verbatim}


\begin{Verbatim}[commandchars=\\\{\}]
{\color{outcolor}Out[{\color{outcolor}21}]:} array([63.81108808, 63.81108808, 63.81108808, 63.81108808, 63.81108808])
\end{Verbatim}
            
    Remember the \texttt{sklearn} dummy regressor?

    \begin{Verbatim}[commandchars=\\\{\}]
{\color{incolor}In [{\color{incolor}22}]:} \PY{n}{y\PYZus{}tr\PYZus{}pred} \PY{o}{=} \PY{n}{dumb\PYZus{}reg}\PY{o}{.}\PY{n}{predict}\PY{p}{(}\PY{n}{X\PYZus{}train}\PY{p}{)}
         \PY{n}{y\PYZus{}tr\PYZus{}pred}\PY{p}{[}\PY{p}{:}\PY{l+m+mi}{5}\PY{p}{]}
\end{Verbatim}


\begin{Verbatim}[commandchars=\\\{\}]
{\color{outcolor}Out[{\color{outcolor}22}]:} array([63.81108808, 63.81108808, 63.81108808, 63.81108808, 63.81108808])
\end{Verbatim}
            
    You can see that \texttt{DummyRegressor} produces exactly the same
results and saves you having to mess about broadcasting the mean (or
whichever other statistic we used - check out the
\href{https://scikit-learn.org/stable/modules/generated/sklearn.dummy.DummyRegressor.html}{documentation}
to see what's available) to an array of the appropriate length. It also
gives you an object with \texttt{fit()} and \texttt{predict()} methods
as well so you can use them as conveniently as any other
\texttt{sklearn} estimator.

    \begin{Verbatim}[commandchars=\\\{\}]
{\color{incolor}In [{\color{incolor}23}]:} \PY{n}{r\PYZus{}squared}\PY{p}{(}\PY{n}{y\PYZus{}train}\PY{p}{,} \PY{n}{y\PYZus{}tr\PYZus{}pred}\PY{p}{)}
\end{Verbatim}


\begin{Verbatim}[commandchars=\\\{\}]
{\color{outcolor}Out[{\color{outcolor}23}]:} 0.0
\end{Verbatim}
            
    Exactly as expected, if you use the average value as your prediction,
you get an \(R^2\) of zero \emph{on our training set}. What if you use
this "model" to predict unseen values from the test set? Remember, of
course, that your "model" is trained on the training set; you still use
the training set mean as your prediction.

    Make your predictions by creating an array of length the size of the
test set with the single value of the (training) mean.

    \begin{Verbatim}[commandchars=\\\{\}]
{\color{incolor}In [{\color{incolor}24}]:} \PY{n}{y\PYZus{}te\PYZus{}pred} \PY{o}{=} \PY{n}{train\PYZus{}mean} \PY{o}{*} \PY{n}{np}\PY{o}{.}\PY{n}{ones}\PY{p}{(}\PY{n+nb}{len}\PY{p}{(}\PY{n}{y\PYZus{}test}\PY{p}{)}\PY{p}{)}
         \PY{n}{r\PYZus{}squared}\PY{p}{(}\PY{n}{y\PYZus{}test}\PY{p}{,} \PY{n}{y\PYZus{}te\PYZus{}pred}\PY{p}{)}
\end{Verbatim}


\begin{Verbatim}[commandchars=\\\{\}]
{\color{outcolor}Out[{\color{outcolor}24}]:} -0.0031235200417913944
\end{Verbatim}
            
    Generally, you can expect performance on a test set to be slightly worse
than on the training set. As you are getting an \(R^2\) of zero on the
training set, there's nowhere to go but negative!

    \(R^2\) is a common metric, and interpretable in terms of the amount of
variance explained, it's less appealing if you want an idea of how
"close" your predictions are to the true values. Metrics that summarise
the difference between predicted and actual values are \emph{mean
absolute error} and \emph{mean squared error}.

    \paragraph{4.7.1.2 Mean Absolute Error}\label{mean-absolute-error}

    This is very simply the average of the absolute errors:

\[MAE = \frac{1}{n}\sum_i^n|y_i - \hat{y}|\]

    \begin{Verbatim}[commandchars=\\\{\}]
{\color{incolor}In [{\color{incolor}25}]:} \PY{c+c1}{\PYZsh{}Code task 7\PYZsh{}}
         \PY{c+c1}{\PYZsh{}Calculate the MAE as defined above}
         \PY{k}{def} \PY{n+nf}{mae}\PY{p}{(}\PY{n}{y}\PY{p}{,} \PY{n}{ypred}\PY{p}{)}\PY{p}{:}
             \PY{l+s+sd}{\PYZdq{}\PYZdq{}\PYZdq{}Mean absolute error.}
         \PY{l+s+sd}{    }
         \PY{l+s+sd}{    Calculate the mean absolute error of the arguments}
         
         \PY{l+s+sd}{    Arguments:}
         \PY{l+s+sd}{    y \PYZhy{}\PYZhy{} the observed values}
         \PY{l+s+sd}{    ypred \PYZhy{}\PYZhy{} the predicted values}
         \PY{l+s+sd}{    \PYZdq{}\PYZdq{}\PYZdq{}}
             \PY{n}{abs\PYZus{}error} \PY{o}{=} \PY{n}{np}\PY{o}{.}\PY{n}{abs}\PY{p}{(}\PY{n}{y} \PY{o}{\PYZhy{}} \PY{n}{ypred}\PY{p}{)}
             \PY{n}{mae} \PY{o}{=} \PY{n}{np}\PY{o}{.}\PY{n}{mean}\PY{p}{(}\PY{n}{abs\PYZus{}error}\PY{p}{)}
             \PY{k}{return} \PY{n}{mae}
\end{Verbatim}


    \begin{Verbatim}[commandchars=\\\{\}]
{\color{incolor}In [{\color{incolor}26}]:} \PY{n}{mae}\PY{p}{(}\PY{n}{y\PYZus{}train}\PY{p}{,} \PY{n}{y\PYZus{}tr\PYZus{}pred}\PY{p}{)}
\end{Verbatim}


\begin{Verbatim}[commandchars=\\\{\}]
{\color{outcolor}Out[{\color{outcolor}26}]:} 17.923463717146785
\end{Verbatim}
            
    \begin{Verbatim}[commandchars=\\\{\}]
{\color{incolor}In [{\color{incolor}27}]:} \PY{n}{mae}\PY{p}{(}\PY{n}{y\PYZus{}test}\PY{p}{,} \PY{n}{y\PYZus{}te\PYZus{}pred}\PY{p}{)}
\end{Verbatim}


\begin{Verbatim}[commandchars=\\\{\}]
{\color{outcolor}Out[{\color{outcolor}27}]:} 19.136142081278486
\end{Verbatim}
            
    Mean absolute error is arguably the most intuitive of all the metrics,
this essentially tells you that, on average, you might expect to be off
by around \textbackslash{}\$19 if you guessed ticket price based on an
average of known values.

    \paragraph{4.7.1.3 Mean Squared Error}\label{mean-squared-error}

    Another common metric (and an important one internally for optimizing
machine learning models) is the mean squared error. This is simply the
average of the square of the errors:

\[MSE = \frac{1}{n}\sum_i^n(y_i - \hat{y})^2\]

    \begin{Verbatim}[commandchars=\\\{\}]
{\color{incolor}In [{\color{incolor}28}]:} \PY{c+c1}{\PYZsh{}Code task 8\PYZsh{}}
         \PY{c+c1}{\PYZsh{}Calculate the MSE as defined above}
         \PY{k}{def} \PY{n+nf}{mse}\PY{p}{(}\PY{n}{y}\PY{p}{,} \PY{n}{ypred}\PY{p}{)}\PY{p}{:}
             \PY{l+s+sd}{\PYZdq{}\PYZdq{}\PYZdq{}Mean square error.}
         \PY{l+s+sd}{    }
         \PY{l+s+sd}{    Calculate the mean square error of the arguments}
         
         \PY{l+s+sd}{    Arguments:}
         \PY{l+s+sd}{    y \PYZhy{}\PYZhy{} the observed values}
         \PY{l+s+sd}{    ypred \PYZhy{}\PYZhy{} the predicted values}
         \PY{l+s+sd}{    \PYZdq{}\PYZdq{}\PYZdq{}}
             \PY{n}{sq\PYZus{}error} \PY{o}{=} \PY{p}{(}\PY{n}{y} \PY{o}{\PYZhy{}} \PY{n}{ypred}\PY{p}{)}\PY{o}{*}\PY{o}{*}\PY{l+m+mi}{2}
             \PY{n}{mse} \PY{o}{=} \PY{n}{np}\PY{o}{.}\PY{n}{mean}\PY{p}{(}\PY{n}{sq\PYZus{}error}\PY{p}{)}
             \PY{k}{return} \PY{n}{mse}
\end{Verbatim}


    \begin{Verbatim}[commandchars=\\\{\}]
{\color{incolor}In [{\color{incolor}29}]:} \PY{n}{mse}\PY{p}{(}\PY{n}{y\PYZus{}train}\PY{p}{,} \PY{n}{y\PYZus{}tr\PYZus{}pred}\PY{p}{)}
\end{Verbatim}


\begin{Verbatim}[commandchars=\\\{\}]
{\color{outcolor}Out[{\color{outcolor}29}]:} 614.1334096969057
\end{Verbatim}
            
    \begin{Verbatim}[commandchars=\\\{\}]
{\color{incolor}In [{\color{incolor}30}]:} \PY{n}{mse}\PY{p}{(}\PY{n}{y\PYZus{}test}\PY{p}{,} \PY{n}{y\PYZus{}te\PYZus{}pred}\PY{p}{)}
\end{Verbatim}


\begin{Verbatim}[commandchars=\\\{\}]
{\color{outcolor}Out[{\color{outcolor}30}]:} 581.4365441953481
\end{Verbatim}
            
    So here, you get a slightly better MSE on the test set than you did on
the train set. And what does a squared error mean anyway? To convert
this back to our measurement space, we often take the square root, to
form the \emph{root mean square error} thus:

    \begin{Verbatim}[commandchars=\\\{\}]
{\color{incolor}In [{\color{incolor}31}]:} \PY{n}{np}\PY{o}{.}\PY{n}{sqrt}\PY{p}{(}\PY{p}{[}\PY{n}{mse}\PY{p}{(}\PY{n}{y\PYZus{}train}\PY{p}{,} \PY{n}{y\PYZus{}tr\PYZus{}pred}\PY{p}{)}\PY{p}{,} \PY{n}{mse}\PY{p}{(}\PY{n}{y\PYZus{}test}\PY{p}{,} \PY{n}{y\PYZus{}te\PYZus{}pred}\PY{p}{)}\PY{p}{]}\PY{p}{)}
\end{Verbatim}


\begin{Verbatim}[commandchars=\\\{\}]
{\color{outcolor}Out[{\color{outcolor}31}]:} array([24.78171523, 24.11299534])
\end{Verbatim}
            
    \subsubsection{4.7.2 sklearn metrics}\label{sklearn-metrics}

    Functions are good, but you don't want to have to define functions every
time we want to assess performance. \texttt{sklearn.metrics} provides
many commonly used metrics, included the ones above.

    \subparagraph{4.7.2.0.1 R-squared}\label{r-squared}

    \begin{Verbatim}[commandchars=\\\{\}]
{\color{incolor}In [{\color{incolor}32}]:} \PY{n}{r2\PYZus{}score}\PY{p}{(}\PY{n}{y\PYZus{}train}\PY{p}{,} \PY{n}{y\PYZus{}tr\PYZus{}pred}\PY{p}{)}\PY{p}{,} \PY{n}{r2\PYZus{}score}\PY{p}{(}\PY{n}{y\PYZus{}test}\PY{p}{,} \PY{n}{y\PYZus{}te\PYZus{}pred}\PY{p}{)}
\end{Verbatim}


\begin{Verbatim}[commandchars=\\\{\}]
{\color{outcolor}Out[{\color{outcolor}32}]:} (0.0, -0.0031235200417913944)
\end{Verbatim}
            
    \subparagraph{4.7.2.0.2 Mean absolute error}\label{mean-absolute-error}

    \begin{Verbatim}[commandchars=\\\{\}]
{\color{incolor}In [{\color{incolor}33}]:} \PY{n}{mean\PYZus{}absolute\PYZus{}error}\PY{p}{(}\PY{n}{y\PYZus{}train}\PY{p}{,} \PY{n}{y\PYZus{}tr\PYZus{}pred}\PY{p}{)}\PY{p}{,} \PY{n}{mean\PYZus{}absolute\PYZus{}error}\PY{p}{(}\PY{n}{y\PYZus{}test}\PY{p}{,} \PY{n}{y\PYZus{}te\PYZus{}pred}\PY{p}{)}
\end{Verbatim}


\begin{Verbatim}[commandchars=\\\{\}]
{\color{outcolor}Out[{\color{outcolor}33}]:} (17.92346371714677, 19.136142081278486)
\end{Verbatim}
            
    \subparagraph{4.7.2.0.3 Mean squared error}\label{mean-squared-error}

    \begin{Verbatim}[commandchars=\\\{\}]
{\color{incolor}In [{\color{incolor}34}]:} \PY{n}{mean\PYZus{}squared\PYZus{}error}\PY{p}{(}\PY{n}{y\PYZus{}train}\PY{p}{,} \PY{n}{y\PYZus{}tr\PYZus{}pred}\PY{p}{)}\PY{p}{,} \PY{n}{mean\PYZus{}squared\PYZus{}error}\PY{p}{(}\PY{n}{y\PYZus{}test}\PY{p}{,} \PY{n}{y\PYZus{}te\PYZus{}pred}\PY{p}{)}
\end{Verbatim}


\begin{Verbatim}[commandchars=\\\{\}]
{\color{outcolor}Out[{\color{outcolor}34}]:} (614.1334096969046, 581.4365441953483)
\end{Verbatim}
            
    \subsubsection{4.7.3 Note On Calculating
Metrics}\label{note-on-calculating-metrics}

    When calling functions to calculate metrics, it is important to take
care in the order of the arguments. Two of the metrics above actually
don't care if the arguments are reversed; one does. Which one cares?

    In a Jupyter code cell, running \texttt{r2\_score?} will bring up the
docstring for the function, and \texttt{r2\_score??} will bring up the
actual code of the function! Try them and compare the source for
\texttt{sklearn}'s function with yours. Feel free to explore what
happens when you reverse the order of the arguments and compare
behaviour of \texttt{sklearn}'s function and yours.

    \begin{Verbatim}[commandchars=\\\{\}]
{\color{incolor}In [{\color{incolor}35}]:} \PY{c+c1}{\PYZsh{} train set \PYZhy{} sklearn}
         \PY{c+c1}{\PYZsh{} correct order, incorrect order}
         \PY{n}{r2\PYZus{}score}\PY{p}{(}\PY{n}{y\PYZus{}train}\PY{p}{,} \PY{n}{y\PYZus{}tr\PYZus{}pred}\PY{p}{)}\PY{p}{,} \PY{n}{r2\PYZus{}score}\PY{p}{(}\PY{n}{y\PYZus{}tr\PYZus{}pred}\PY{p}{,} \PY{n}{y\PYZus{}train}\PY{p}{)}
\end{Verbatim}


\begin{Verbatim}[commandchars=\\\{\}]
{\color{outcolor}Out[{\color{outcolor}35}]:} (0.0, -3.041041349306602e+30)
\end{Verbatim}
            
    \begin{Verbatim}[commandchars=\\\{\}]
{\color{incolor}In [{\color{incolor}36}]:} \PY{c+c1}{\PYZsh{} test set \PYZhy{} sklearn}
         \PY{c+c1}{\PYZsh{} correct order, incorrect order}
         \PY{n}{r2\PYZus{}score}\PY{p}{(}\PY{n}{y\PYZus{}test}\PY{p}{,} \PY{n}{y\PYZus{}te\PYZus{}pred}\PY{p}{)}\PY{p}{,} \PY{n}{r2\PYZus{}score}\PY{p}{(}\PY{n}{y\PYZus{}te\PYZus{}pred}\PY{p}{,} \PY{n}{y\PYZus{}test}\PY{p}{)}
\end{Verbatim}


\begin{Verbatim}[commandchars=\\\{\}]
{\color{outcolor}Out[{\color{outcolor}36}]:} (-0.0031235200417913944, 0.0)
\end{Verbatim}
            
    \begin{Verbatim}[commandchars=\\\{\}]
{\color{incolor}In [{\color{incolor}37}]:} \PY{c+c1}{\PYZsh{} train set \PYZhy{} using our homebrew function}
         \PY{c+c1}{\PYZsh{} correct order, incorrect order}
         \PY{n}{r\PYZus{}squared}\PY{p}{(}\PY{n}{y\PYZus{}train}\PY{p}{,} \PY{n}{y\PYZus{}tr\PYZus{}pred}\PY{p}{)}\PY{p}{,} \PY{n}{r\PYZus{}squared}\PY{p}{(}\PY{n}{y\PYZus{}tr\PYZus{}pred}\PY{p}{,} \PY{n}{y\PYZus{}train}\PY{p}{)}
\end{Verbatim}


\begin{Verbatim}[commandchars=\\\{\}]
{\color{outcolor}Out[{\color{outcolor}37}]:} (0.0, -3.041041349306602e+30)
\end{Verbatim}
            
    \begin{Verbatim}[commandchars=\\\{\}]
{\color{incolor}In [{\color{incolor}38}]:} \PY{c+c1}{\PYZsh{} test set \PYZhy{} using our homebrew function}
         \PY{c+c1}{\PYZsh{} correct order, incorrect order}
         \PY{n}{r\PYZus{}squared}\PY{p}{(}\PY{n}{y\PYZus{}test}\PY{p}{,} \PY{n}{y\PYZus{}te\PYZus{}pred}\PY{p}{)}\PY{p}{,} \PY{n}{r\PYZus{}squared}\PY{p}{(}\PY{n}{y\PYZus{}te\PYZus{}pred}\PY{p}{,} \PY{n}{y\PYZus{}test}\PY{p}{)}
\end{Verbatim}


    \begin{Verbatim}[commandchars=\\\{\}]
C:\textbackslash{}Users\textbackslash{}gyhazarika\textbackslash{}AppData\textbackslash{}Local\textbackslash{}Continuum\textbackslash{}anaconda3\textbackslash{}lib\textbackslash{}site-packages\textbackslash{}ipykernel\_launcher.py:15: RuntimeWarning: divide by zero encountered in double\_scalars
  from ipykernel import kernelapp as app

    \end{Verbatim}

\begin{Verbatim}[commandchars=\\\{\}]
{\color{outcolor}Out[{\color{outcolor}38}]:} (-0.0031235200417913944, -inf)
\end{Verbatim}
            
    You can get very different results swapping the argument order. It's
worth highlighting this because data scientists do this too much in the
real world! Don't be one of them! Frequently the argument order doesn't
matter, but it will bite you when you do it with a function that does
care. It's sloppy, bad practice and if you don't make a habit of putting
arguments in the right order, you will forget!

Remember: * argument order matters, * check function syntax with
\texttt{func?} in a code cell

    \subsection{4.8 Initial Models}\label{initial-models}

    \subsubsection{4.8.1 Imputing missing feature (predictor)
values}\label{imputing-missing-feature-predictor-values}

    Recall when performing EDA, you imputed (filled in) some missing values
in pandas. You did this judiciously for exploratory/visualization
purposes. You left many missing values in the data. You can impute
missing values using scikit-learn, but note that you should learn values
to impute from a train split and apply that to the test split to then
assess how well your imputation worked.

    \paragraph{4.8.1.1 Impute missing values with
median}\label{impute-missing-values-with-median}

    There's missing values. Recall from your data exploration that many
distributions were skewed. Your first thought might be to impute missing
values using the median.

    \subparagraph{4.8.1.1.1 Learn the values to impute from the train
set}\label{learn-the-values-to-impute-from-the-train-set}

    \begin{Verbatim}[commandchars=\\\{\}]
{\color{incolor}In [{\color{incolor}39}]:} \PY{c+c1}{\PYZsh{} These are the values we\PYZsq{}ll use to fill in any missing values}
         \PY{n}{X\PYZus{}defaults\PYZus{}median} \PY{o}{=} \PY{n}{X\PYZus{}train}\PY{o}{.}\PY{n}{median}\PY{p}{(}\PY{p}{)}
         \PY{n}{X\PYZus{}defaults\PYZus{}median}
\end{Verbatim}


\begin{Verbatim}[commandchars=\\\{\}]
{\color{outcolor}Out[{\color{outcolor}39}]:} summit\_elev                           2215.000000
         vertical\_drop                          750.000000
         base\_elev                             1300.000000
         trams                                    0.000000
         fastSixes                                0.000000
         fastQuads                                0.000000
         quad                                     1.000000
         triple                                   1.000000
         double                                   1.000000
         surface                                  2.000000
         total\_chairs                             7.000000
         Runs                                    28.000000
         TerrainParks                             2.000000
         LongestRun\_mi                            1.000000
         SkiableTerrain\_ac                      170.000000
         Snow Making\_ac                          96.500000
         daysOpenLastYear                       109.000000
         yearsOpen                               57.000000
         averageSnowfall                        120.000000
         projectedDaysOpen                      115.000000
         NightSkiing\_ac                          70.000000
         resorts\_per\_state                       14.000000
         resorts\_per\_100kcapita                   0.240449
         resorts\_per\_100ksq\_mile                 16.508461
         resort\_skiable\_area\_ac\_state\_ratio       0.064103
         resort\_days\_open\_state\_ratio             0.081710
         resort\_terrain\_park\_state\_ratio          0.090909
         resort\_night\_skiing\_state\_ratio          0.087805
         total\_chairs\_runs\_ratio                  0.200000
         total\_chairs\_skiable\_ratio               0.040323
         fastQuads\_runs\_ratio                     0.000000
         fastQuads\_skiable\_ratio                  0.000000
         dtype: float64
\end{Verbatim}
            
    \subparagraph{4.8.1.1.2 Apply the imputation to both train and test
splits}\label{apply-the-imputation-to-both-train-and-test-splits}

    \begin{Verbatim}[commandchars=\\\{\}]
{\color{incolor}In [{\color{incolor}40}]:} \PY{c+c1}{\PYZsh{}Code task 9\PYZsh{}}
         \PY{c+c1}{\PYZsh{}Call `X\PYZus{}train` and `X\PYZus{}test`\PYZsq{}s `fillna()` method, passing `X\PYZus{}defaults\PYZus{}median` as the values to use}
         \PY{c+c1}{\PYZsh{}Assign the results to `X\PYZus{}tr` and `X\PYZus{}te`, respectively}
         \PY{n}{X\PYZus{}tr} \PY{o}{=} \PY{n}{X\PYZus{}train}\PY{o}{.}\PY{n}{fillna}\PY{p}{(}\PY{n}{X\PYZus{}defaults\PYZus{}median}\PY{p}{)}
         \PY{n}{X\PYZus{}te} \PY{o}{=} \PY{n}{X\PYZus{}test}\PY{o}{.}\PY{n}{fillna}\PY{p}{(}\PY{n}{X\PYZus{}defaults\PYZus{}median}\PY{p}{)}
\end{Verbatim}


    \subparagraph{4.8.1.1.3 Scale the data}\label{scale-the-data}

    As you have features measured in many different units, with numbers that
vary by orders of magnitude, start off by scaling them to put them all
on a consistent scale. The
\href{https://scikit-learn.org/stable/modules/generated/sklearn.preprocessing.StandardScaler.html}{StandardScaler}
scales each feature to zero mean and unit variance.

    \begin{Verbatim}[commandchars=\\\{\}]
{\color{incolor}In [{\color{incolor}43}]:} \PY{c+c1}{\PYZsh{}Code task 10\PYZsh{}}
         \PY{c+c1}{\PYZsh{}Call the StandardScaler`s fit method on `X\PYZus{}tr` to fit the scaler}
         \PY{c+c1}{\PYZsh{}then use it\PYZsq{}s `transform()` method to apply the scaling to both the train and test split}
         \PY{c+c1}{\PYZsh{}data (`X\PYZus{}tr` and `X\PYZus{}te`), naming the results `X\PYZus{}tr\PYZus{}scaled` and `X\PYZus{}te\PYZus{}scaled`, respectively}
         \PY{n}{scaler} \PY{o}{=} \PY{n}{StandardScaler}\PY{p}{(}\PY{p}{)}
         \PY{n}{scaler}\PY{o}{.}\PY{n}{fit}\PY{p}{(}\PY{n}{X\PYZus{}tr}\PY{p}{)}
         \PY{n}{X\PYZus{}tr\PYZus{}scaled} \PY{o}{=} \PY{n}{scaler}\PY{o}{.}\PY{n}{transform}\PY{p}{(}\PY{n}{X\PYZus{}tr}\PY{p}{)}
         \PY{n}{X\PYZus{}te\PYZus{}scaled} \PY{o}{=} \PY{n}{scaler}\PY{o}{.}\PY{n}{transform}\PY{p}{(}\PY{n}{X\PYZus{}te}\PY{p}{)}
\end{Verbatim}


    \subparagraph{4.8.1.1.4 Train the model on the train
split}\label{train-the-model-on-the-train-split}

    \begin{Verbatim}[commandchars=\\\{\}]
{\color{incolor}In [{\color{incolor}44}]:} \PY{n}{lm} \PY{o}{=} \PY{n}{LinearRegression}\PY{p}{(}\PY{p}{)}\PY{o}{.}\PY{n}{fit}\PY{p}{(}\PY{n}{X\PYZus{}tr\PYZus{}scaled}\PY{p}{,} \PY{n}{y\PYZus{}train}\PY{p}{)}
\end{Verbatim}


    \subparagraph{4.8.1.1.5 Make predictions using the model on both train
and test
splits}\label{make-predictions-using-the-model-on-both-train-and-test-splits}

    \begin{Verbatim}[commandchars=\\\{\}]
{\color{incolor}In [{\color{incolor}45}]:} \PY{c+c1}{\PYZsh{}Code task 11\PYZsh{}}
         \PY{c+c1}{\PYZsh{}Call the `predict()` method of the model (`lm`) on both the (scaled) train and test data}
         \PY{c+c1}{\PYZsh{}Assign the predictions to `y\PYZus{}tr\PYZus{}pred` and `y\PYZus{}te\PYZus{}pred`, respectively}
         \PY{n}{y\PYZus{}tr\PYZus{}pred} \PY{o}{=} \PY{n}{lm}\PY{o}{.}\PY{n}{predict}\PY{p}{(}\PY{n}{X\PYZus{}tr\PYZus{}scaled}\PY{p}{)}
         \PY{n}{y\PYZus{}te\PYZus{}pred} \PY{o}{=} \PY{n}{lm}\PY{o}{.}\PY{n}{predict}\PY{p}{(}\PY{n}{X\PYZus{}te\PYZus{}scaled}\PY{p}{)}
\end{Verbatim}


    \subparagraph{4.8.1.1.6 Assess model
performance}\label{assess-model-performance}

    \begin{Verbatim}[commandchars=\\\{\}]
{\color{incolor}In [{\color{incolor}46}]:} \PY{c+c1}{\PYZsh{} r\PYZca{}2 \PYZhy{} train, test}
         \PY{n}{median\PYZus{}r2} \PY{o}{=} \PY{n}{r2\PYZus{}score}\PY{p}{(}\PY{n}{y\PYZus{}train}\PY{p}{,} \PY{n}{y\PYZus{}tr\PYZus{}pred}\PY{p}{)}\PY{p}{,} \PY{n}{r2\PYZus{}score}\PY{p}{(}\PY{n}{y\PYZus{}test}\PY{p}{,} \PY{n}{y\PYZus{}te\PYZus{}pred}\PY{p}{)}
         \PY{n}{median\PYZus{}r2}
\end{Verbatim}


\begin{Verbatim}[commandchars=\\\{\}]
{\color{outcolor}Out[{\color{outcolor}46}]:} (0.8202944748175386, 0.7127828086489635)
\end{Verbatim}
            
    Recall that you estimated ticket price by simply using a known average.
As expected, this produced an \(R^2\) of zero for both the training and
test set, because \(R^2\) tells us how much of the variance you're
explaining beyond that of using just the mean, and you were using just
the mean. Here we see that our simple linear regression model explains
over 80\% of the variance on the train set and over 70\% on the test
set. Clearly you are onto something, although the much lower value for
the test set suggests you're overfitting somewhat. This isn't a surprise
as you've made no effort to select a parsimonious set of features or
deal with multicollinearity in our data.

    \begin{Verbatim}[commandchars=\\\{\}]
{\color{incolor}In [{\color{incolor}48}]:} \PY{c+c1}{\PYZsh{}Code task 12\PYZsh{}}
         \PY{c+c1}{\PYZsh{}Now calculate the mean absolute error scores using `sklearn`\PYZsq{}s `mean\PYZus{}absolute\PYZus{}error` function}
         \PY{c+c1}{\PYZsh{} as we did above for R\PYZca{}2}
         \PY{c+c1}{\PYZsh{} MAE \PYZhy{} train, test}
         \PY{n}{median\PYZus{}mae} \PY{o}{=} \PY{n}{mean\PYZus{}absolute\PYZus{}error}\PY{p}{(}\PY{n}{y\PYZus{}train}\PY{p}{,} \PY{n}{y\PYZus{}tr\PYZus{}pred}\PY{p}{)}\PY{p}{,} \PY{n}{mean\PYZus{}absolute\PYZus{}error}\PY{p}{(}\PY{n}{y\PYZus{}test}\PY{p}{,} \PY{n}{y\PYZus{}te\PYZus{}pred}\PY{p}{)}
         \PY{n}{median\PYZus{}mae}
\end{Verbatim}


\begin{Verbatim}[commandchars=\\\{\}]
{\color{outcolor}Out[{\color{outcolor}48}]:} (8.518515955068468, 9.441831036848503)
\end{Verbatim}
            
    Using this model, then, on average you'd expect to estimate a ticket
price within \textbackslash{}\$9 or so of the real price. This is much,
much better than the \textbackslash{}\$19 from just guessing using the
average. There may be something to this machine learning lark after all!

    \begin{Verbatim}[commandchars=\\\{\}]
{\color{incolor}In [{\color{incolor}49}]:} \PY{c+c1}{\PYZsh{}Code task 13\PYZsh{}}
         \PY{c+c1}{\PYZsh{}And also do the same using `sklearn`\PYZsq{}s `mean\PYZus{}squared\PYZus{}error`}
         \PY{c+c1}{\PYZsh{} MSE \PYZhy{} train, test}
         \PY{n}{median\PYZus{}mse} \PY{o}{=} \PY{n}{mean\PYZus{}squared\PYZus{}error}\PY{p}{(}\PY{n}{y\PYZus{}train}\PY{p}{,} \PY{n}{y\PYZus{}tr\PYZus{}pred}\PY{p}{)}\PY{p}{,} \PY{n}{mean\PYZus{}squared\PYZus{}error}\PY{p}{(}\PY{n}{y\PYZus{}test}\PY{p}{,} \PY{n}{y\PYZus{}te\PYZus{}pred}\PY{p}{)}
         \PY{n}{median\PYZus{}mse}
\end{Verbatim}


\begin{Verbatim}[commandchars=\\\{\}]
{\color{outcolor}Out[{\color{outcolor}49}]:} (110.363166921678, 166.4785720164187)
\end{Verbatim}
            
    \paragraph{4.8.1.2 Impute missing values with the
mean}\label{impute-missing-values-with-the-mean}

    You chose to use the median for filling missing values because of the
skew of many of our predictor feature distributions. What if you wanted
to try something else, such as the mean?

    \subparagraph{4.8.1.2.1 Learn the values to impute from the train
set}\label{learn-the-values-to-impute-from-the-train-set}

    \begin{Verbatim}[commandchars=\\\{\}]
{\color{incolor}In [{\color{incolor}50}]:} \PY{c+c1}{\PYZsh{}Code task 14\PYZsh{}}
         \PY{c+c1}{\PYZsh{}As we did for the median above, calculate mean values for imputing missing values}
         \PY{c+c1}{\PYZsh{} These are the values we\PYZsq{}ll use to fill in any missing values}
         \PY{n}{X\PYZus{}defaults\PYZus{}mean} \PY{o}{=} \PY{n}{X\PYZus{}train}\PY{o}{.}\PY{n}{mean}\PY{p}{(}\PY{p}{)}
         \PY{n}{X\PYZus{}defaults\PYZus{}mean}
\end{Verbatim}


\begin{Verbatim}[commandchars=\\\{\}]
{\color{outcolor}Out[{\color{outcolor}50}]:} summit\_elev                           4074.554404
         vertical\_drop                         1043.196891
         base\_elev                             3020.512953
         trams                                    0.103627
         fastSixes                                0.072539
         fastQuads                                0.673575
         quad                                     1.010363
         triple                                   1.440415
         double                                   1.813472
         surface                                  2.497409
         total\_chairs                             7.611399
         Runs                                    41.188482
         TerrainParks                             2.434783
         LongestRun\_mi                            1.293122
         SkiableTerrain\_ac                      448.785340
         Snow Making\_ac                         129.601190
         daysOpenLastYear                       110.100629
         yearsOpen                               56.559585
         averageSnowfall                        162.310160
         projectedDaysOpen                      115.920245
         NightSkiing\_ac                          86.384615
         resorts\_per\_state                       14.518135
         resorts\_per\_100kcapita                   0.351899
         resorts\_per\_100ksq\_mile                 35.690324
         resort\_skiable\_area\_ac\_state\_ratio       0.109497
         resort\_days\_open\_state\_ratio             0.144236
         resort\_terrain\_park\_state\_ratio          0.134344
         resort\_night\_skiing\_state\_ratio          0.174531
         total\_chairs\_runs\_ratio                  0.271441
         total\_chairs\_skiable\_ratio               0.070483
         fastQuads\_runs\_ratio                     0.010401
         fastQuads\_skiable\_ratio                  0.001633
         dtype: float64
\end{Verbatim}
            
    By eye, you can immediately tell that your replacement values are much
higher than those from using the median.

    \subparagraph{4.8.1.2.2 Apply the imputation to both train and test
splits}\label{apply-the-imputation-to-both-train-and-test-splits}

    \begin{Verbatim}[commandchars=\\\{\}]
{\color{incolor}In [{\color{incolor}51}]:} \PY{n}{X\PYZus{}tr} \PY{o}{=} \PY{n}{X\PYZus{}train}\PY{o}{.}\PY{n}{fillna}\PY{p}{(}\PY{n}{X\PYZus{}defaults\PYZus{}mean}\PY{p}{)}
         \PY{n}{X\PYZus{}te} \PY{o}{=} \PY{n}{X\PYZus{}test}\PY{o}{.}\PY{n}{fillna}\PY{p}{(}\PY{n}{X\PYZus{}defaults\PYZus{}mean}\PY{p}{)}
\end{Verbatim}


    \subparagraph{4.8.1.2.3 Scale the data}\label{scale-the-data}

    \begin{Verbatim}[commandchars=\\\{\}]
{\color{incolor}In [{\color{incolor}52}]:} \PY{n}{scaler} \PY{o}{=} \PY{n}{StandardScaler}\PY{p}{(}\PY{p}{)}
         \PY{n}{scaler}\PY{o}{.}\PY{n}{fit}\PY{p}{(}\PY{n}{X\PYZus{}tr}\PY{p}{)}
         \PY{n}{X\PYZus{}tr\PYZus{}scaled} \PY{o}{=} \PY{n}{scaler}\PY{o}{.}\PY{n}{transform}\PY{p}{(}\PY{n}{X\PYZus{}tr}\PY{p}{)}
         \PY{n}{X\PYZus{}te\PYZus{}scaled} \PY{o}{=} \PY{n}{scaler}\PY{o}{.}\PY{n}{transform}\PY{p}{(}\PY{n}{X\PYZus{}te}\PY{p}{)}
\end{Verbatim}


    \subparagraph{4.8.1.2.4 Train the model on the train
split}\label{train-the-model-on-the-train-split}

    \begin{Verbatim}[commandchars=\\\{\}]
{\color{incolor}In [{\color{incolor}53}]:} \PY{n}{lm} \PY{o}{=} \PY{n}{LinearRegression}\PY{p}{(}\PY{p}{)}\PY{o}{.}\PY{n}{fit}\PY{p}{(}\PY{n}{X\PYZus{}tr\PYZus{}scaled}\PY{p}{,} \PY{n}{y\PYZus{}train}\PY{p}{)}
\end{Verbatim}


    \subparagraph{4.8.1.2.5 Make predictions using the model on both train
and test
splits}\label{make-predictions-using-the-model-on-both-train-and-test-splits}

    \begin{Verbatim}[commandchars=\\\{\}]
{\color{incolor}In [{\color{incolor}54}]:} \PY{n}{y\PYZus{}tr\PYZus{}pred} \PY{o}{=} \PY{n}{lm}\PY{o}{.}\PY{n}{predict}\PY{p}{(}\PY{n}{X\PYZus{}tr\PYZus{}scaled}\PY{p}{)}
         \PY{n}{y\PYZus{}te\PYZus{}pred} \PY{o}{=} \PY{n}{lm}\PY{o}{.}\PY{n}{predict}\PY{p}{(}\PY{n}{X\PYZus{}te\PYZus{}scaled}\PY{p}{)}
\end{Verbatim}


    \subparagraph{4.8.1.2.6 Assess model
performance}\label{assess-model-performance}

    \begin{Verbatim}[commandchars=\\\{\}]
{\color{incolor}In [{\color{incolor}55}]:} \PY{n}{r2\PYZus{}score}\PY{p}{(}\PY{n}{y\PYZus{}train}\PY{p}{,} \PY{n}{y\PYZus{}tr\PYZus{}pred}\PY{p}{)}\PY{p}{,} \PY{n}{r2\PYZus{}score}\PY{p}{(}\PY{n}{y\PYZus{}test}\PY{p}{,} \PY{n}{y\PYZus{}te\PYZus{}pred}\PY{p}{)}
\end{Verbatim}


\begin{Verbatim}[commandchars=\\\{\}]
{\color{outcolor}Out[{\color{outcolor}55}]:} (0.8193701248472637, 0.7098970868386896)
\end{Verbatim}
            
    \begin{Verbatim}[commandchars=\\\{\}]
{\color{incolor}In [{\color{incolor}56}]:} \PY{n}{mean\PYZus{}absolute\PYZus{}error}\PY{p}{(}\PY{n}{y\PYZus{}train}\PY{p}{,} \PY{n}{y\PYZus{}tr\PYZus{}pred}\PY{p}{)}\PY{p}{,} \PY{n}{mean\PYZus{}absolute\PYZus{}error}\PY{p}{(}\PY{n}{y\PYZus{}test}\PY{p}{,} \PY{n}{y\PYZus{}te\PYZus{}pred}\PY{p}{)}
\end{Verbatim}


\begin{Verbatim}[commandchars=\\\{\}]
{\color{outcolor}Out[{\color{outcolor}56}]:} (8.541326227293274, 9.438374036159757)
\end{Verbatim}
            
    \begin{Verbatim}[commandchars=\\\{\}]
{\color{incolor}In [{\color{incolor}57}]:} \PY{n}{mean\PYZus{}squared\PYZus{}error}\PY{p}{(}\PY{n}{y\PYZus{}train}\PY{p}{,} \PY{n}{y\PYZus{}tr\PYZus{}pred}\PY{p}{)}\PY{p}{,} \PY{n}{mean\PYZus{}squared\PYZus{}error}\PY{p}{(}\PY{n}{y\PYZus{}test}\PY{p}{,} \PY{n}{y\PYZus{}te\PYZus{}pred}\PY{p}{)}
\end{Verbatim}


\begin{Verbatim}[commandchars=\\\{\}]
{\color{outcolor}Out[{\color{outcolor}57}]:} (110.93084112067612, 168.15121161000027)
\end{Verbatim}
            
    These results don't seem very different to when you used the median for
imputing missing values. Perhaps it doesn't make much difference here.
Maybe your overtraining dominates. Maybe other feature transformations,
such as taking the log, would help. You could try with just a subset of
features rather than using all of them as inputs.

To perform the median/mean comparison, you copied and pasted a lot of
code just to change the function for imputing missing values. It would
make more sense to write a function that performed the sequence of
steps: 1. impute missing values 2. scale the features 3. train a model
4. calculate model performance

But these are common steps and \texttt{sklearn} provides something much
better than writing custom functions.

    \subsubsection{4.8.2 Pipelines}\label{pipelines}

    One of the most important and useful components of \texttt{sklearn} is
the
\href{https://scikit-learn.org/stable/modules/generated/sklearn.pipeline.Pipeline.html}{pipeline}.
In place of \texttt{panda}'s \texttt{fillna} DataFrame method, there is
\texttt{sklearn}'s \texttt{SimpleImputer}. Remember the first linear
model above performed the steps:

\begin{enumerate}
\def\labelenumi{\arabic{enumi}.}
\tightlist
\item
  replace missing values with the median for each feature
\item
  scale the data to zero mean and unit variance
\item
  train a linear regression model
\end{enumerate}

and all these steps were trained on the train split and then applied to
the test split for assessment.

The pipeline below defines exactly those same steps. Crucially, the
resultant \texttt{Pipeline} object has a \texttt{fit()} method and a
\texttt{predict()} method, just like the \texttt{LinearRegression()}
object itself. Just as you might create a linear regression model and
train it with \texttt{.fit()} and predict with \texttt{.predict()}, you
can wrap the entire process of imputing and feature scaling and
regression in a single object you can train with \texttt{.fit()} and
predict with \texttt{.predict()}. And that's basically a pipeline: a
model on steroids.

    \paragraph{4.8.2.1 Define the pipeline}\label{define-the-pipeline}

    \begin{Verbatim}[commandchars=\\\{\}]
{\color{incolor}In [{\color{incolor}58}]:} \PY{n}{pipe} \PY{o}{=} \PY{n}{make\PYZus{}pipeline}\PY{p}{(}
             \PY{n}{SimpleImputer}\PY{p}{(}\PY{n}{strategy}\PY{o}{=}\PY{l+s+s1}{\PYZsq{}}\PY{l+s+s1}{median}\PY{l+s+s1}{\PYZsq{}}\PY{p}{)}\PY{p}{,} 
             \PY{n}{StandardScaler}\PY{p}{(}\PY{p}{)}\PY{p}{,} 
             \PY{n}{LinearRegression}\PY{p}{(}\PY{p}{)}
         \PY{p}{)}
\end{Verbatim}


    \begin{Verbatim}[commandchars=\\\{\}]
{\color{incolor}In [{\color{incolor}59}]:} \PY{n+nb}{type}\PY{p}{(}\PY{n}{pipe}\PY{p}{)}
\end{Verbatim}


\begin{Verbatim}[commandchars=\\\{\}]
{\color{outcolor}Out[{\color{outcolor}59}]:} sklearn.pipeline.Pipeline
\end{Verbatim}
            
    \begin{Verbatim}[commandchars=\\\{\}]
{\color{incolor}In [{\color{incolor}60}]:} \PY{n+nb}{hasattr}\PY{p}{(}\PY{n}{pipe}\PY{p}{,} \PY{l+s+s1}{\PYZsq{}}\PY{l+s+s1}{fit}\PY{l+s+s1}{\PYZsq{}}\PY{p}{)}\PY{p}{,} \PY{n+nb}{hasattr}\PY{p}{(}\PY{n}{pipe}\PY{p}{,} \PY{l+s+s1}{\PYZsq{}}\PY{l+s+s1}{predict}\PY{l+s+s1}{\PYZsq{}}\PY{p}{)}
\end{Verbatim}


\begin{Verbatim}[commandchars=\\\{\}]
{\color{outcolor}Out[{\color{outcolor}60}]:} (True, True)
\end{Verbatim}
            
    \paragraph{4.8.2.2 Fit the pipeline}\label{fit-the-pipeline}

    Here, a single call to the pipeline's \texttt{fit()} method combines the
steps of learning the imputation (determining what values to use to fill
the missing ones), the scaling (determining the mean to subtract and the
variance to divide by), and then training the model. It does this all in
the one call with the training data as arguments.

    \begin{Verbatim}[commandchars=\\\{\}]
{\color{incolor}In [{\color{incolor}62}]:} \PY{c+c1}{\PYZsh{}Code task 15\PYZsh{}}
         \PY{c+c1}{\PYZsh{}Call the pipe\PYZsq{}s `fit()` method with `X\PYZus{}train` and `y\PYZus{}train` as arguments}
         \PY{n}{pipe}\PY{o}{.}\PY{n}{fit}\PY{p}{(}\PY{n}{X\PYZus{}train}\PY{p}{,} \PY{n}{y\PYZus{}train}\PY{p}{)}
\end{Verbatim}


\begin{Verbatim}[commandchars=\\\{\}]
{\color{outcolor}Out[{\color{outcolor}62}]:} Pipeline(steps=[('simpleimputer', SimpleImputer(strategy='median')),
                         ('standardscaler', StandardScaler()),
                         ('linearregression', LinearRegression())])
\end{Verbatim}
            
    \paragraph{4.8.2.3 Make predictions on the train and test
sets}\label{make-predictions-on-the-train-and-test-sets}

    \begin{Verbatim}[commandchars=\\\{\}]
{\color{incolor}In [{\color{incolor}63}]:} \PY{n}{y\PYZus{}tr\PYZus{}pred} \PY{o}{=} \PY{n}{pipe}\PY{o}{.}\PY{n}{predict}\PY{p}{(}\PY{n}{X\PYZus{}train}\PY{p}{)}
         \PY{n}{y\PYZus{}te\PYZus{}pred} \PY{o}{=} \PY{n}{pipe}\PY{o}{.}\PY{n}{predict}\PY{p}{(}\PY{n}{X\PYZus{}test}\PY{p}{)}
\end{Verbatim}


    \paragraph{4.8.2.4 Assess performance}\label{assess-performance}

    \begin{Verbatim}[commandchars=\\\{\}]
{\color{incolor}In [{\color{incolor}64}]:} \PY{n}{r2\PYZus{}score}\PY{p}{(}\PY{n}{y\PYZus{}train}\PY{p}{,} \PY{n}{y\PYZus{}tr\PYZus{}pred}\PY{p}{)}\PY{p}{,} \PY{n}{r2\PYZus{}score}\PY{p}{(}\PY{n}{y\PYZus{}test}\PY{p}{,} \PY{n}{y\PYZus{}te\PYZus{}pred}\PY{p}{)}
\end{Verbatim}


\begin{Verbatim}[commandchars=\\\{\}]
{\color{outcolor}Out[{\color{outcolor}64}]:} (0.8202944748175386, 0.7127828086489635)
\end{Verbatim}
            
    And compare with your earlier (non-pipeline) result:

    \begin{Verbatim}[commandchars=\\\{\}]
{\color{incolor}In [{\color{incolor}65}]:} \PY{n}{median\PYZus{}r2}
\end{Verbatim}


\begin{Verbatim}[commandchars=\\\{\}]
{\color{outcolor}Out[{\color{outcolor}65}]:} (0.8202944748175386, 0.7127828086489635)
\end{Verbatim}
            
    \begin{Verbatim}[commandchars=\\\{\}]
{\color{incolor}In [{\color{incolor}66}]:} \PY{n}{mean\PYZus{}absolute\PYZus{}error}\PY{p}{(}\PY{n}{y\PYZus{}train}\PY{p}{,} \PY{n}{y\PYZus{}tr\PYZus{}pred}\PY{p}{)}\PY{p}{,} \PY{n}{mean\PYZus{}absolute\PYZus{}error}\PY{p}{(}\PY{n}{y\PYZus{}test}\PY{p}{,} \PY{n}{y\PYZus{}te\PYZus{}pred}\PY{p}{)}
\end{Verbatim}


\begin{Verbatim}[commandchars=\\\{\}]
{\color{outcolor}Out[{\color{outcolor}66}]:} (8.518515955068468, 9.441831036848503)
\end{Verbatim}
            
    \begin{Verbatim}[commandchars=\\\{\}]
{\color{incolor}In [{\color{incolor}67}]:} \PY{n}{Compare} \PY{k}{with} \PY{n}{your} \PY{n}{earlier} \PY{n}{result}\PY{p}{:}
\end{Verbatim}


    \begin{Verbatim}[commandchars=\\\{\}]

          File "<ipython-input-67-441b87b2c3ea>", line 1
        Compare with your earlier result:
                   \^{}
    SyntaxError: invalid syntax
    

    \end{Verbatim}

    \begin{Verbatim}[commandchars=\\\{\}]
{\color{incolor}In [{\color{incolor}69}]:} \PY{n}{median\PYZus{}mae}
\end{Verbatim}


\begin{Verbatim}[commandchars=\\\{\}]
{\color{outcolor}Out[{\color{outcolor}69}]:} (8.518515955068468, 9.441831036848503)
\end{Verbatim}
            
    \begin{Verbatim}[commandchars=\\\{\}]
{\color{incolor}In [{\color{incolor}70}]:} \PY{n}{mean\PYZus{}squared\PYZus{}error}\PY{p}{(}\PY{n}{y\PYZus{}train}\PY{p}{,} \PY{n}{y\PYZus{}tr\PYZus{}pred}\PY{p}{)}\PY{p}{,} \PY{n}{mean\PYZus{}squared\PYZus{}error}\PY{p}{(}\PY{n}{y\PYZus{}test}\PY{p}{,} \PY{n}{y\PYZus{}te\PYZus{}pred}\PY{p}{)}
\end{Verbatim}


\begin{Verbatim}[commandchars=\\\{\}]
{\color{outcolor}Out[{\color{outcolor}70}]:} (110.363166921678, 166.4785720164187)
\end{Verbatim}
            
    Compare with your earlier result:

    \begin{Verbatim}[commandchars=\\\{\}]
{\color{incolor}In [{\color{incolor}71}]:} \PY{n}{median\PYZus{}mse}
\end{Verbatim}


\begin{Verbatim}[commandchars=\\\{\}]
{\color{outcolor}Out[{\color{outcolor}71}]:} (110.363166921678, 166.4785720164187)
\end{Verbatim}
            
    These results confirm the pipeline is doing exactly what's expected, and
results are identical to your earlier steps. This allows you to move
faster but with confidence.

    \subsection{4.9 Refining The Linear
Model}\label{refining-the-linear-model}

    You suspected the model was overfitting. This is no real surprise given
the number of features you blindly used. It's likely a judicious subset
of features would generalize better. \texttt{sklearn} has a number of
feature selection functions available. The one you'll use here is
\texttt{SelectKBest} which, as you might guess, selects the k best
features. You can read about SelectKBest
\href{https://scikit-learn.org/stable/modules/generated/sklearn.feature_selection.SelectKBest.html\#sklearn.feature_selection.SelectKBest}{here}.
\texttt{f\_regression} is just the
\href{https://scikit-learn.org/stable/modules/generated/sklearn.feature_selection.f_regression.html\#sklearn.feature_selection.f_regression}{score
function} you're using because you're performing regression. It's
important to choose an appropriate one for your machine learning task.

    \subsubsection{4.9.1 Define the pipeline}\label{define-the-pipeline}

    Redefine your pipeline to include this feature selection step:

    \begin{Verbatim}[commandchars=\\\{\}]
{\color{incolor}In [{\color{incolor}72}]:} \PY{c+c1}{\PYZsh{}Code task 16\PYZsh{}}
         \PY{c+c1}{\PYZsh{}Add `SelectKBest` as a step in the pipeline between `StandardScaler()` and `LinearRegression()`}
         \PY{c+c1}{\PYZsh{}Don\PYZsq{}t forget to tell it to use `f\PYZus{}regression` as its score function}
         \PY{n}{pipe} \PY{o}{=} \PY{n}{make\PYZus{}pipeline}\PY{p}{(}
             \PY{n}{SimpleImputer}\PY{p}{(}\PY{n}{strategy}\PY{o}{=}\PY{l+s+s1}{\PYZsq{}}\PY{l+s+s1}{median}\PY{l+s+s1}{\PYZsq{}}\PY{p}{)}\PY{p}{,} 
             \PY{n}{StandardScaler}\PY{p}{(}\PY{p}{)}\PY{p}{,}
             \PY{n}{SelectKBest}\PY{p}{(}\PY{n}{f\PYZus{}regression}\PY{p}{)}\PY{p}{,}
             \PY{n}{LinearRegression}\PY{p}{(}\PY{p}{)}
         \PY{p}{)}
\end{Verbatim}


    \subsubsection{4.9.2 Fit the pipeline}\label{fit-the-pipeline}

    \begin{Verbatim}[commandchars=\\\{\}]
{\color{incolor}In [{\color{incolor}73}]:} \PY{n}{pipe}\PY{o}{.}\PY{n}{fit}\PY{p}{(}\PY{n}{X\PYZus{}train}\PY{p}{,} \PY{n}{y\PYZus{}train}\PY{p}{)}
\end{Verbatim}


\begin{Verbatim}[commandchars=\\\{\}]
{\color{outcolor}Out[{\color{outcolor}73}]:} Pipeline(steps=[('simpleimputer', SimpleImputer(strategy='median')),
                         ('standardscaler', StandardScaler()),
                         ('selectkbest',
                          SelectKBest(score\_func=<function f\_regression at 0x000001BDFFB16598>)),
                         ('linearregression', LinearRegression())])
\end{Verbatim}
            
    \subsubsection{4.9.3 Assess performance on the train and test
set}\label{assess-performance-on-the-train-and-test-set}

    \begin{Verbatim}[commandchars=\\\{\}]
{\color{incolor}In [{\color{incolor}74}]:} \PY{n}{y\PYZus{}tr\PYZus{}pred} \PY{o}{=} \PY{n}{pipe}\PY{o}{.}\PY{n}{predict}\PY{p}{(}\PY{n}{X\PYZus{}train}\PY{p}{)}
         \PY{n}{y\PYZus{}te\PYZus{}pred} \PY{o}{=} \PY{n}{pipe}\PY{o}{.}\PY{n}{predict}\PY{p}{(}\PY{n}{X\PYZus{}test}\PY{p}{)}
\end{Verbatim}


    \begin{Verbatim}[commandchars=\\\{\}]
{\color{incolor}In [{\color{incolor}75}]:} \PY{n}{r2\PYZus{}score}\PY{p}{(}\PY{n}{y\PYZus{}train}\PY{p}{,} \PY{n}{y\PYZus{}tr\PYZus{}pred}\PY{p}{)}\PY{p}{,} \PY{n}{r2\PYZus{}score}\PY{p}{(}\PY{n}{y\PYZus{}test}\PY{p}{,} \PY{n}{y\PYZus{}te\PYZus{}pred}\PY{p}{)}
\end{Verbatim}


\begin{Verbatim}[commandchars=\\\{\}]
{\color{outcolor}Out[{\color{outcolor}75}]:} (0.7674914326052744, 0.6259877354190834)
\end{Verbatim}
            
    \begin{Verbatim}[commandchars=\\\{\}]
{\color{incolor}In [{\color{incolor}76}]:} \PY{n}{mean\PYZus{}absolute\PYZus{}error}\PY{p}{(}\PY{n}{y\PYZus{}train}\PY{p}{,} \PY{n}{y\PYZus{}tr\PYZus{}pred}\PY{p}{)}\PY{p}{,} \PY{n}{mean\PYZus{}absolute\PYZus{}error}\PY{p}{(}\PY{n}{y\PYZus{}test}\PY{p}{,} \PY{n}{y\PYZus{}te\PYZus{}pred}\PY{p}{)}
\end{Verbatim}


\begin{Verbatim}[commandchars=\\\{\}]
{\color{outcolor}Out[{\color{outcolor}76}]:} (9.501495079727485, 11.201830190332057)
\end{Verbatim}
            
    This has made things worse! Clearly selecting a subset of features has
an impact on performance. \texttt{SelectKBest} defaults to k=10. You've
just seen that 10 is worse than using all features. What is the best k?
You could create a new pipeline with a different value of k:

    \subsubsection{4.9.4 Define a new pipeline to select a different number
of
features}\label{define-a-new-pipeline-to-select-a-different-number-of-features}

    \begin{Verbatim}[commandchars=\\\{\}]
{\color{incolor}In [{\color{incolor}78}]:} \PY{c+c1}{\PYZsh{}Code task 17\PYZsh{}}
         \PY{c+c1}{\PYZsh{}Modify the `SelectKBest` step to use a value of 15 for k}
         \PY{n}{pipe15} \PY{o}{=} \PY{n}{make\PYZus{}pipeline}\PY{p}{(}
             \PY{n}{SimpleImputer}\PY{p}{(}\PY{n}{strategy}\PY{o}{=}\PY{l+s+s1}{\PYZsq{}}\PY{l+s+s1}{median}\PY{l+s+s1}{\PYZsq{}}\PY{p}{)}\PY{p}{,} 
             \PY{n}{StandardScaler}\PY{p}{(}\PY{p}{)}\PY{p}{,}
             \PY{n}{SelectKBest}\PY{p}{(}\PY{n}{f\PYZus{}regression}\PY{p}{,} \PY{n}{k}\PY{o}{=}\PY{l+m+mi}{15}\PY{p}{)}\PY{p}{,}
             \PY{n}{LinearRegression}\PY{p}{(}\PY{p}{)}
         \PY{p}{)}
\end{Verbatim}


    \subsubsection{4.9.5 Fit the pipeline}\label{fit-the-pipeline}

    \begin{Verbatim}[commandchars=\\\{\}]
{\color{incolor}In [{\color{incolor}79}]:} \PY{n}{pipe15}\PY{o}{.}\PY{n}{fit}\PY{p}{(}\PY{n}{X\PYZus{}train}\PY{p}{,} \PY{n}{y\PYZus{}train}\PY{p}{)}
\end{Verbatim}


\begin{Verbatim}[commandchars=\\\{\}]
{\color{outcolor}Out[{\color{outcolor}79}]:} Pipeline(steps=[('simpleimputer', SimpleImputer(strategy='median')),
                         ('standardscaler', StandardScaler()),
                         ('selectkbest',
                          SelectKBest(k=15,
                                      score\_func=<function f\_regression at 0x000001BDFFB16598>)),
                         ('linearregression', LinearRegression())])
\end{Verbatim}
            
    \subsubsection{4.9.6 Assess performance on train and test
data}\label{assess-performance-on-train-and-test-data}

    \begin{Verbatim}[commandchars=\\\{\}]
{\color{incolor}In [{\color{incolor}80}]:} \PY{n}{y\PYZus{}tr\PYZus{}pred} \PY{o}{=} \PY{n}{pipe15}\PY{o}{.}\PY{n}{predict}\PY{p}{(}\PY{n}{X\PYZus{}train}\PY{p}{)}
         \PY{n}{y\PYZus{}te\PYZus{}pred} \PY{o}{=} \PY{n}{pipe15}\PY{o}{.}\PY{n}{predict}\PY{p}{(}\PY{n}{X\PYZus{}test}\PY{p}{)}
\end{Verbatim}


    \begin{Verbatim}[commandchars=\\\{\}]
{\color{incolor}In [{\color{incolor}81}]:} \PY{n}{r2\PYZus{}score}\PY{p}{(}\PY{n}{y\PYZus{}train}\PY{p}{,} \PY{n}{y\PYZus{}tr\PYZus{}pred}\PY{p}{)}\PY{p}{,} \PY{n}{r2\PYZus{}score}\PY{p}{(}\PY{n}{y\PYZus{}test}\PY{p}{,} \PY{n}{y\PYZus{}te\PYZus{}pred}\PY{p}{)}
\end{Verbatim}


\begin{Verbatim}[commandchars=\\\{\}]
{\color{outcolor}Out[{\color{outcolor}81}]:} (0.7924096060483825, 0.6376199973170795)
\end{Verbatim}
            
    \begin{Verbatim}[commandchars=\\\{\}]
{\color{incolor}In [{\color{incolor}82}]:} \PY{n}{mean\PYZus{}absolute\PYZus{}error}\PY{p}{(}\PY{n}{y\PYZus{}train}\PY{p}{,} \PY{n}{y\PYZus{}tr\PYZus{}pred}\PY{p}{)}\PY{p}{,} \PY{n}{mean\PYZus{}absolute\PYZus{}error}\PY{p}{(}\PY{n}{y\PYZus{}test}\PY{p}{,} \PY{n}{y\PYZus{}te\PYZus{}pred}\PY{p}{)}
\end{Verbatim}


\begin{Verbatim}[commandchars=\\\{\}]
{\color{outcolor}Out[{\color{outcolor}82}]:} (9.211767769307114, 10.488246867294356)
\end{Verbatim}
            
    You could keep going, trying different values of k, training a model,
measuring performance on the test set, and then picking the model with
the best test set performance. There's a fundamental problem with this
approach: \emph{you're tuning the model to the arbitrary test set}! If
you continue this way you'll end up with a model works well on the
particular quirks of our test set \emph{but fails to generalize to new
data}. The whole point of keeping a test set is for it to be a set of
that new data, to check how well our model might perform on data it
hasn't seen.

The way around this is a technique called \emph{cross-validation}. You
partition the training set into k folds, train our model on k-1 of those
folds, and calculate performance on the fold not used in training. This
procedure then cycles through k times with a different fold held back
each time. Thus you end up building k models on k sets of data with k
estimates of how the model performs on unseen data but without having to
touch the test set.

    \subsubsection{4.9.7 Assessing performance using
cross-validation}\label{assessing-performance-using-cross-validation}

    \begin{Verbatim}[commandchars=\\\{\}]
{\color{incolor}In [{\color{incolor}83}]:} \PY{n}{cv\PYZus{}results} \PY{o}{=} \PY{n}{cross\PYZus{}validate}\PY{p}{(}\PY{n}{pipe15}\PY{p}{,} \PY{n}{X\PYZus{}train}\PY{p}{,} \PY{n}{y\PYZus{}train}\PY{p}{,} \PY{n}{cv}\PY{o}{=}\PY{l+m+mi}{5}\PY{p}{)}
\end{Verbatim}


    \begin{Verbatim}[commandchars=\\\{\}]
{\color{incolor}In [{\color{incolor}84}]:} \PY{n}{cv\PYZus{}scores} \PY{o}{=} \PY{n}{cv\PYZus{}results}\PY{p}{[}\PY{l+s+s1}{\PYZsq{}}\PY{l+s+s1}{test\PYZus{}score}\PY{l+s+s1}{\PYZsq{}}\PY{p}{]}
         \PY{n}{cv\PYZus{}scores}
\end{Verbatim}


\begin{Verbatim}[commandchars=\\\{\}]
{\color{outcolor}Out[{\color{outcolor}84}]:} array([0.63760862, 0.72831381, 0.74443537, 0.5487915 , 0.50441472])
\end{Verbatim}
            
    Without using the same random state for initializing the CV folds, your
actual numbers will be different.

    \begin{Verbatim}[commandchars=\\\{\}]
{\color{incolor}In [{\color{incolor}85}]:} \PY{n}{np}\PY{o}{.}\PY{n}{mean}\PY{p}{(}\PY{n}{cv\PYZus{}scores}\PY{p}{)}\PY{p}{,} \PY{n}{np}\PY{o}{.}\PY{n}{std}\PY{p}{(}\PY{n}{cv\PYZus{}scores}\PY{p}{)}
\end{Verbatim}


\begin{Verbatim}[commandchars=\\\{\}]
{\color{outcolor}Out[{\color{outcolor}85}]:} (0.6327128053007863, 0.09502487849877701)
\end{Verbatim}
            
    These results highlight that assessing model performance in inherently
open to variability. You'll get different results depending on the
quirks of which points are in which fold. An advantage of this is that
you can also obtain an estimate of the variability, or uncertainty, in
your performance estimate.

    \begin{Verbatim}[commandchars=\\\{\}]
{\color{incolor}In [{\color{incolor}86}]:} \PY{n}{np}\PY{o}{.}\PY{n}{round}\PY{p}{(}\PY{p}{(}\PY{n}{np}\PY{o}{.}\PY{n}{mean}\PY{p}{(}\PY{n}{cv\PYZus{}scores}\PY{p}{)} \PY{o}{\PYZhy{}} \PY{l+m+mi}{2} \PY{o}{*} \PY{n}{np}\PY{o}{.}\PY{n}{std}\PY{p}{(}\PY{n}{cv\PYZus{}scores}\PY{p}{)}\PY{p}{,} \PY{n}{np}\PY{o}{.}\PY{n}{mean}\PY{p}{(}\PY{n}{cv\PYZus{}scores}\PY{p}{)} \PY{o}{+} \PY{l+m+mi}{2} \PY{o}{*} \PY{n}{np}\PY{o}{.}\PY{n}{std}\PY{p}{(}\PY{n}{cv\PYZus{}scores}\PY{p}{)}\PY{p}{)}\PY{p}{,} \PY{l+m+mi}{2}\PY{p}{)}
\end{Verbatim}


\begin{Verbatim}[commandchars=\\\{\}]
{\color{outcolor}Out[{\color{outcolor}86}]:} array([0.44, 0.82])
\end{Verbatim}
            
    \subsubsection{4.9.8 Hyperparameter search using
GridSearchCV}\label{hyperparameter-search-using-gridsearchcv}

    Pulling the above together, we have: * a pipeline that * imputes missing
values * scales the data * selects the k best features * trains a linear
regression model * a technique (cross-validation) for estimating model
performance

Now you want to use cross-validation for multiple values of k and use
cross-validation to pick the value of k that gives the best performance.
\texttt{make\_pipeline} automatically names each step as the lowercase
name of the step and the parameters of the step are then accessed by
appending a double underscore followed by the parameter name. You know
the name of the step will be 'selectkbest' and you know the parameter is
'k'.

You can also list the names of all the parameters in a pipeline like
this:

    \begin{Verbatim}[commandchars=\\\{\}]
{\color{incolor}In [{\color{incolor}88}]:} \PY{c+c1}{\PYZsh{}Code task 18\PYZsh{}}
         \PY{c+c1}{\PYZsh{}Call `pipe`\PYZsq{}s `get\PYZus{}params()` method to get a dict of available parameters and print their names}
         \PY{c+c1}{\PYZsh{}using dict\PYZsq{}s `keys()` method}
         \PY{n}{pipe}\PY{o}{.}\PY{n}{get\PYZus{}params}\PY{p}{(}\PY{p}{)}\PY{o}{.}\PY{n}{keys}\PY{p}{(}\PY{p}{)}
\end{Verbatim}


\begin{Verbatim}[commandchars=\\\{\}]
{\color{outcolor}Out[{\color{outcolor}88}]:} dict\_keys(['memory', 'steps', 'verbose', 'simpleimputer', 'standardscaler', 'selectkbest', 'linearregression', 'simpleimputer\_\_add\_indicator', 'simpleimputer\_\_copy', 'simpleimputer\_\_fill\_value', 'simpleimputer\_\_missing\_values', 'simpleimputer\_\_strategy', 'simpleimputer\_\_verbose', 'standardscaler\_\_copy', 'standardscaler\_\_with\_mean', 'standardscaler\_\_with\_std', 'selectkbest\_\_k', 'selectkbest\_\_score\_func', 'linearregression\_\_copy\_X', 'linearregression\_\_fit\_intercept', 'linearregression\_\_n\_jobs', 'linearregression\_\_normalize', 'linearregression\_\_positive'])
\end{Verbatim}
            
    The above can be particularly useful as your pipelines becomes more
complex (you can even nest pipelines within pipelines).

    \begin{Verbatim}[commandchars=\\\{\}]
{\color{incolor}In [{\color{incolor}89}]:} \PY{n}{k} \PY{o}{=} \PY{p}{[}\PY{n}{k}\PY{o}{+}\PY{l+m+mi}{1} \PY{k}{for} \PY{n}{k} \PY{o+ow}{in} \PY{n+nb}{range}\PY{p}{(}\PY{n+nb}{len}\PY{p}{(}\PY{n}{X\PYZus{}train}\PY{o}{.}\PY{n}{columns}\PY{p}{)}\PY{p}{)}\PY{p}{]}
         \PY{n}{grid\PYZus{}params} \PY{o}{=} \PY{p}{\PYZob{}}\PY{l+s+s1}{\PYZsq{}}\PY{l+s+s1}{selectkbest\PYZus{}\PYZus{}k}\PY{l+s+s1}{\PYZsq{}}\PY{p}{:} \PY{n}{k}\PY{p}{\PYZcb{}}
\end{Verbatim}


    Now you have a range of \texttt{k} to investigate. Is 1 feature best? 2?
3? 4? All of them? You could write a for loop and iterate over each
possible value, doing all the housekeeping oyurselves to track the best
value of k. But this is a common task so there's a built in function in
\texttt{sklearn}. This is
\href{https://scikit-learn.org/stable/modules/generated/sklearn.model_selection.GridSearchCV.html}{\texttt{GridSearchCV}}.
This takes the pipeline object, in fact it takes anything with a
\texttt{.fit()} and \texttt{.predict()} method. In simple cases with no
feature selection or imputation or feature scaling etc. you may see the
classifier or regressor object itself directly passed into
\texttt{GridSearchCV}. The other key input is the parameters and values
to search over. Optional parameters include the cross-validation
strategy and number of CPUs to use.

    \begin{Verbatim}[commandchars=\\\{\}]
{\color{incolor}In [{\color{incolor}90}]:} \PY{n}{lr\PYZus{}grid\PYZus{}cv} \PY{o}{=} \PY{n}{GridSearchCV}\PY{p}{(}\PY{n}{pipe}\PY{p}{,} \PY{n}{param\PYZus{}grid}\PY{o}{=}\PY{n}{grid\PYZus{}params}\PY{p}{,} \PY{n}{cv}\PY{o}{=}\PY{l+m+mi}{5}\PY{p}{,} \PY{n}{n\PYZus{}jobs}\PY{o}{=}\PY{o}{\PYZhy{}}\PY{l+m+mi}{1}\PY{p}{)}
\end{Verbatim}


    \begin{Verbatim}[commandchars=\\\{\}]
{\color{incolor}In [{\color{incolor}91}]:} \PY{n}{lr\PYZus{}grid\PYZus{}cv}\PY{o}{.}\PY{n}{fit}\PY{p}{(}\PY{n}{X\PYZus{}train}\PY{p}{,} \PY{n}{y\PYZus{}train}\PY{p}{)}
\end{Verbatim}


\begin{Verbatim}[commandchars=\\\{\}]
{\color{outcolor}Out[{\color{outcolor}91}]:} GridSearchCV(cv=5,
                      estimator=Pipeline(steps=[('simpleimputer',
                                                 SimpleImputer(strategy='median')),
                                                ('standardscaler', StandardScaler()),
                                                ('selectkbest',
                                                 SelectKBest(score\_func=<function f\_regression at 0x000001BDFFB16598>)),
                                                ('linearregression',
                                                 LinearRegression())]),
                      n\_jobs=-1,
                      param\_grid=\{'selectkbest\_\_k': [1, 2, 3, 4, 5, 6, 7, 8, 9, 10, 11,
                                                     12, 13, 14, 15, 16, 17, 18, 19, 20,
                                                     21, 22, 23, 24, 25, 26, 27, 28, 29,
                                                     30, {\ldots}]\})
\end{Verbatim}
            
    \begin{Verbatim}[commandchars=\\\{\}]
{\color{incolor}In [{\color{incolor}92}]:} \PY{n}{score\PYZus{}mean} \PY{o}{=} \PY{n}{lr\PYZus{}grid\PYZus{}cv}\PY{o}{.}\PY{n}{cv\PYZus{}results\PYZus{}}\PY{p}{[}\PY{l+s+s1}{\PYZsq{}}\PY{l+s+s1}{mean\PYZus{}test\PYZus{}score}\PY{l+s+s1}{\PYZsq{}}\PY{p}{]}
         \PY{n}{score\PYZus{}std} \PY{o}{=} \PY{n}{lr\PYZus{}grid\PYZus{}cv}\PY{o}{.}\PY{n}{cv\PYZus{}results\PYZus{}}\PY{p}{[}\PY{l+s+s1}{\PYZsq{}}\PY{l+s+s1}{std\PYZus{}test\PYZus{}score}\PY{l+s+s1}{\PYZsq{}}\PY{p}{]}
         \PY{n}{cv\PYZus{}k} \PY{o}{=} \PY{p}{[}\PY{n}{k} \PY{k}{for} \PY{n}{k} \PY{o+ow}{in} \PY{n}{lr\PYZus{}grid\PYZus{}cv}\PY{o}{.}\PY{n}{cv\PYZus{}results\PYZus{}}\PY{p}{[}\PY{l+s+s1}{\PYZsq{}}\PY{l+s+s1}{param\PYZus{}selectkbest\PYZus{}\PYZus{}k}\PY{l+s+s1}{\PYZsq{}}\PY{p}{]}\PY{p}{]}
\end{Verbatim}


    \begin{Verbatim}[commandchars=\\\{\}]
{\color{incolor}In [{\color{incolor}94}]:} \PY{c+c1}{\PYZsh{}Code task 19\PYZsh{}}
         \PY{c+c1}{\PYZsh{}Print the `best\PYZus{}params\PYZus{}` attribute of `lr\PYZus{}grid\PYZus{}cv`}
         \PY{n}{lr\PYZus{}grid\PYZus{}cv}\PY{o}{.}\PY{n}{best\PYZus{}params\PYZus{}}
\end{Verbatim}


\begin{Verbatim}[commandchars=\\\{\}]
{\color{outcolor}Out[{\color{outcolor}94}]:} \{'selectkbest\_\_k': 8\}
\end{Verbatim}
            
    \begin{Verbatim}[commandchars=\\\{\}]
{\color{incolor}In [{\color{incolor}95}]:} \PY{c+c1}{\PYZsh{}Code task 20\PYZsh{}}
         \PY{c+c1}{\PYZsh{}Assign the value of k from the above dict of `best\PYZus{}params\PYZus{}` and assign it to `best\PYZus{}k`}
         \PY{n}{best\PYZus{}k} \PY{o}{=} \PY{n}{lr\PYZus{}grid\PYZus{}cv}\PY{o}{.}\PY{n}{best\PYZus{}params\PYZus{}}\PY{p}{[}\PY{l+s+s1}{\PYZsq{}}\PY{l+s+s1}{selectkbest\PYZus{}\PYZus{}k}\PY{l+s+s1}{\PYZsq{}}\PY{p}{]}
         \PY{n}{plt}\PY{o}{.}\PY{n}{subplots}\PY{p}{(}\PY{n}{figsize}\PY{o}{=}\PY{p}{(}\PY{l+m+mi}{10}\PY{p}{,} \PY{l+m+mi}{5}\PY{p}{)}\PY{p}{)}
         \PY{n}{plt}\PY{o}{.}\PY{n}{errorbar}\PY{p}{(}\PY{n}{cv\PYZus{}k}\PY{p}{,} \PY{n}{score\PYZus{}mean}\PY{p}{,} \PY{n}{yerr}\PY{o}{=}\PY{n}{score\PYZus{}std}\PY{p}{)}
         \PY{n}{plt}\PY{o}{.}\PY{n}{axvline}\PY{p}{(}\PY{n}{x}\PY{o}{=}\PY{n}{best\PYZus{}k}\PY{p}{,} \PY{n}{c}\PY{o}{=}\PY{l+s+s1}{\PYZsq{}}\PY{l+s+s1}{r}\PY{l+s+s1}{\PYZsq{}}\PY{p}{,} \PY{n}{ls}\PY{o}{=}\PY{l+s+s1}{\PYZsq{}}\PY{l+s+s1}{\PYZhy{}\PYZhy{}}\PY{l+s+s1}{\PYZsq{}}\PY{p}{,} \PY{n}{alpha}\PY{o}{=}\PY{o}{.}\PY{l+m+mi}{5}\PY{p}{)}
         \PY{n}{plt}\PY{o}{.}\PY{n}{xlabel}\PY{p}{(}\PY{l+s+s1}{\PYZsq{}}\PY{l+s+s1}{k}\PY{l+s+s1}{\PYZsq{}}\PY{p}{)}
         \PY{n}{plt}\PY{o}{.}\PY{n}{ylabel}\PY{p}{(}\PY{l+s+s1}{\PYZsq{}}\PY{l+s+s1}{CV score (r\PYZhy{}squared)}\PY{l+s+s1}{\PYZsq{}}\PY{p}{)}
         \PY{n}{plt}\PY{o}{.}\PY{n}{title}\PY{p}{(}\PY{l+s+s1}{\PYZsq{}}\PY{l+s+s1}{Pipeline mean CV score (error bars +/\PYZhy{} 1sd)}\PY{l+s+s1}{\PYZsq{}}\PY{p}{)}\PY{p}{;}
\end{Verbatim}


    \begin{center}
    \adjustimage{max size={0.9\linewidth}{0.9\paperheight}}{output_179_0.png}
    \end{center}
    { \hspace*{\fill} \\}
    
    The above suggests a good value for k is 8. There was an initial rapid
increase with k, followed by a slow decline. Also noticeable is the
variance of the results greatly increase above k=8. As you increasingly
overfit, expect greater swings in performance as different points move
in and out of the train/test folds.

    Which features were most useful? Step into your best model, shown below.
Starting with the fitted grid search object, you get the best estimator,
then the named step 'selectkbest', for which you can its
\texttt{get\_support()} method for a logical mask of the features
selected.

    \begin{Verbatim}[commandchars=\\\{\}]
{\color{incolor}In [{\color{incolor}96}]:} \PY{n}{selected} \PY{o}{=} \PY{n}{lr\PYZus{}grid\PYZus{}cv}\PY{o}{.}\PY{n}{best\PYZus{}estimator\PYZus{}}\PY{o}{.}\PY{n}{named\PYZus{}steps}\PY{o}{.}\PY{n}{selectkbest}\PY{o}{.}\PY{n}{get\PYZus{}support}\PY{p}{(}\PY{p}{)}
\end{Verbatim}


    Similarly, instead of using the 'selectkbest' named step, you can access
the named step for the linear regression model and, from that, grab the
model coefficients via its \texttt{coef\_} attribute:

    \begin{Verbatim}[commandchars=\\\{\}]
{\color{incolor}In [{\color{incolor}97}]:} \PY{c+c1}{\PYZsh{}Code task 21\PYZsh{}}
         \PY{c+c1}{\PYZsh{}Get the linear model coefficients from the `coef\PYZus{}` attribute and store in `coefs`,}
         \PY{c+c1}{\PYZsh{}get the matching feature names from the column names of the dataframe,}
         \PY{c+c1}{\PYZsh{}and display the results as a pandas Series with `coefs` as the values and `features` as the index,}
         \PY{c+c1}{\PYZsh{}sorting the values in descending order}
         \PY{n}{coefs} \PY{o}{=} \PY{n}{lr\PYZus{}grid\PYZus{}cv}\PY{o}{.}\PY{n}{best\PYZus{}estimator\PYZus{}}\PY{o}{.}\PY{n}{named\PYZus{}steps}\PY{o}{.}\PY{n}{linearregression}\PY{o}{.}\PY{n}{coef\PYZus{}}
         \PY{n}{features} \PY{o}{=} \PY{n}{X\PYZus{}train}\PY{o}{.}\PY{n}{columns}\PY{p}{[}\PY{n}{selected}\PY{p}{]}
         \PY{n}{pd}\PY{o}{.}\PY{n}{Series}\PY{p}{(}\PY{n}{coefs}\PY{p}{,} \PY{n}{index}\PY{o}{=}\PY{n}{features}\PY{p}{)}\PY{o}{.}\PY{n}{sort\PYZus{}values}\PY{p}{(}\PY{n}{ascending}\PY{o}{=}\PY{k+kc}{False}\PY{p}{)}
\end{Verbatim}


\begin{Verbatim}[commandchars=\\\{\}]
{\color{outcolor}Out[{\color{outcolor}97}]:} vertical\_drop        10.767857
         Snow Making\_ac        6.290074
         total\_chairs          5.794156
         fastQuads             5.745626
         Runs                  5.370555
         LongestRun\_mi         0.181814
         trams                -4.142024
         SkiableTerrain\_ac    -5.249780
         dtype: float64
\end{Verbatim}
            
    These results suggest that vertical drop is your biggest positive
feature. This makes intuitive sense and is consistent with what you saw
during the EDA work. Also, you see the area covered by snow making
equipment is a strong positive as well. People like guaranteed skiing!
The skiable terrain area is negatively associated with ticket price!
This seems odd. People will pay less for larger resorts? There could be
all manner of reasons for this. It could be an effect whereby larger
resorts can host more visitors at any one time and so can charge less
per ticket. As has been mentioned previously, the data are missing
information about visitor numbers. Bear in mind, the coefficient for
skiable terrain is negative \emph{for this model}. For example, if you
kept the total number of chairs and fastQuads constant, but increased
the skiable terrain extent, you might imagine the resort is worse off
because the chairlift capacity is stretched thinner.

    \subsection{4.10 Random Forest Model}\label{random-forest-model}

    A model that can work very well in a lot of cases is the random forest.
For regression, this is provided by \texttt{sklearn}'s
\texttt{RandomForestRegressor} class.

Time to stop the bad practice of repeatedly checking performance on the
test split. Instead, go straight from defining the pipeline to assessing
performance using cross-validation. \texttt{cross\_validate} will
perform the fitting as part of the process. This uses the default
settings for the random forest so you'll then proceed to investigate
some different hyperparameters.

    \subsubsection{4.10.1 Define the pipeline}\label{define-the-pipeline}

    \begin{Verbatim}[commandchars=\\\{\}]
{\color{incolor}In [{\color{incolor}99}]:} \PY{c+c1}{\PYZsh{}Code task 22\PYZsh{}}
         \PY{c+c1}{\PYZsh{}Define a pipeline comprising the steps:}
         \PY{c+c1}{\PYZsh{}SimpleImputer() with a strategy of \PYZsq{}median\PYZsq{}}
         \PY{c+c1}{\PYZsh{}StandardScaler(),}
         \PY{c+c1}{\PYZsh{}and then RandomForestRegressor() with a random state of 47}
         \PY{n}{RF\PYZus{}pipe} \PY{o}{=} \PY{n}{make\PYZus{}pipeline}\PY{p}{(}
             \PY{n}{SimpleImputer}\PY{p}{(}\PY{n}{strategy}\PY{o}{=}\PY{l+s+s1}{\PYZsq{}}\PY{l+s+s1}{median}\PY{l+s+s1}{\PYZsq{}}\PY{p}{)}\PY{p}{,}
             \PY{n}{StandardScaler}\PY{p}{(}\PY{p}{)}\PY{p}{,}
             \PY{n}{RandomForestRegressor}\PY{p}{(}\PY{n}{random\PYZus{}state}\PY{o}{=}\PY{l+m+mi}{47}\PY{p}{)}
         \PY{p}{)}
\end{Verbatim}


    \subsubsection{4.10.2 Fit and assess performance using
cross-validation}\label{fit-and-assess-performance-using-cross-validation}

    \begin{Verbatim}[commandchars=\\\{\}]
{\color{incolor}In [{\color{incolor}100}]:} \PY{c+c1}{\PYZsh{}Code task 23\PYZsh{}}
          \PY{c+c1}{\PYZsh{}Call `cross\PYZus{}validate` to estimate the pipeline\PYZsq{}s performance.}
          \PY{c+c1}{\PYZsh{}Pass it the random forest pipe object, `X\PYZus{}train` and `y\PYZus{}train`,}
          \PY{c+c1}{\PYZsh{}and get it to use 5\PYZhy{}fold cross\PYZhy{}validation}
          \PY{n}{rf\PYZus{}default\PYZus{}cv\PYZus{}results} \PY{o}{=} \PY{n}{cross\PYZus{}validate}\PY{p}{(}\PY{n}{RF\PYZus{}pipe}\PY{p}{,} \PY{n}{X\PYZus{}train}\PY{p}{,} \PY{n}{y\PYZus{}train}\PY{p}{,} \PY{n}{cv}\PY{o}{=}\PY{l+m+mi}{5}\PY{p}{)}
\end{Verbatim}


    \begin{Verbatim}[commandchars=\\\{\}]
{\color{incolor}In [{\color{incolor}101}]:} \PY{n}{rf\PYZus{}cv\PYZus{}scores} \PY{o}{=} \PY{n}{rf\PYZus{}default\PYZus{}cv\PYZus{}results}\PY{p}{[}\PY{l+s+s1}{\PYZsq{}}\PY{l+s+s1}{test\PYZus{}score}\PY{l+s+s1}{\PYZsq{}}\PY{p}{]}
          \PY{n}{rf\PYZus{}cv\PYZus{}scores}
\end{Verbatim}


\begin{Verbatim}[commandchars=\\\{\}]
{\color{outcolor}Out[{\color{outcolor}101}]:} array([0.68168453, 0.80364275, 0.77803903, 0.6088032 , 0.62602249])
\end{Verbatim}
            
    \begin{Verbatim}[commandchars=\\\{\}]
{\color{incolor}In [{\color{incolor}102}]:} \PY{n}{np}\PY{o}{.}\PY{n}{mean}\PY{p}{(}\PY{n}{rf\PYZus{}cv\PYZus{}scores}\PY{p}{)}\PY{p}{,} \PY{n}{np}\PY{o}{.}\PY{n}{std}\PY{p}{(}\PY{n}{rf\PYZus{}cv\PYZus{}scores}\PY{p}{)}
\end{Verbatim}


\begin{Verbatim}[commandchars=\\\{\}]
{\color{outcolor}Out[{\color{outcolor}102}]:} (0.6996383982492244, 0.07868448354957026)
\end{Verbatim}
            
    \subsubsection{4.10.3 Hyperparameter search using
GridSearchCV}\label{hyperparameter-search-using-gridsearchcv}

    Random forest has a number of hyperparameters that can be explored,
however here you'll limit yourselves to exploring some different values
for the number of trees. You'll try it with and without feature scaling,
and try both the mean and median as strategies for imputing missing
values.

    \begin{Verbatim}[commandchars=\\\{\}]
{\color{incolor}In [{\color{incolor}103}]:} \PY{n}{n\PYZus{}est} \PY{o}{=} \PY{p}{[}\PY{n+nb}{int}\PY{p}{(}\PY{n}{n}\PY{p}{)} \PY{k}{for} \PY{n}{n} \PY{o+ow}{in} \PY{n}{np}\PY{o}{.}\PY{n}{logspace}\PY{p}{(}\PY{n}{start}\PY{o}{=}\PY{l+m+mi}{1}\PY{p}{,} \PY{n}{stop}\PY{o}{=}\PY{l+m+mi}{3}\PY{p}{,} \PY{n}{num}\PY{o}{=}\PY{l+m+mi}{20}\PY{p}{)}\PY{p}{]}
          \PY{n}{grid\PYZus{}params} \PY{o}{=} \PY{p}{\PYZob{}}
                  \PY{l+s+s1}{\PYZsq{}}\PY{l+s+s1}{randomforestregressor\PYZus{}\PYZus{}n\PYZus{}estimators}\PY{l+s+s1}{\PYZsq{}}\PY{p}{:} \PY{n}{n\PYZus{}est}\PY{p}{,}
                  \PY{l+s+s1}{\PYZsq{}}\PY{l+s+s1}{standardscaler}\PY{l+s+s1}{\PYZsq{}}\PY{p}{:} \PY{p}{[}\PY{n}{StandardScaler}\PY{p}{(}\PY{p}{)}\PY{p}{,} \PY{k+kc}{None}\PY{p}{]}\PY{p}{,}
                  \PY{l+s+s1}{\PYZsq{}}\PY{l+s+s1}{simpleimputer\PYZus{}\PYZus{}strategy}\PY{l+s+s1}{\PYZsq{}}\PY{p}{:} \PY{p}{[}\PY{l+s+s1}{\PYZsq{}}\PY{l+s+s1}{mean}\PY{l+s+s1}{\PYZsq{}}\PY{p}{,} \PY{l+s+s1}{\PYZsq{}}\PY{l+s+s1}{median}\PY{l+s+s1}{\PYZsq{}}\PY{p}{]}
          \PY{p}{\PYZcb{}}
          \PY{n}{grid\PYZus{}params}
\end{Verbatim}


\begin{Verbatim}[commandchars=\\\{\}]
{\color{outcolor}Out[{\color{outcolor}103}]:} \{'randomforestregressor\_\_n\_estimators': [10,
            12,
            16,
            20,
            26,
            33,
            42,
            54,
            69,
            88,
            112,
            143,
            183,
            233,
            297,
            379,
            483,
            615,
            784,
            1000],
           'standardscaler': [StandardScaler(), None],
           'simpleimputer\_\_strategy': ['mean', 'median']\}
\end{Verbatim}
            
    \begin{Verbatim}[commandchars=\\\{\}]
{\color{incolor}In [{\color{incolor}109}]:} \PY{c+c1}{\PYZsh{}Code task 24\PYZsh{}}
          \PY{c+c1}{\PYZsh{}Call `GridSearchCV` with the random forest pipeline, passing in the above `grid\PYZus{}params`}
          \PY{c+c1}{\PYZsh{}dict for parameters to evaluate, 5\PYZhy{}fold cross\PYZhy{}validation, and all available CPU cores (if desired)}
          \PY{n}{rf\PYZus{}grid\PYZus{}cv} \PY{o}{=} \PY{n}{GridSearchCV}\PY{p}{(}\PY{n}{RF\PYZus{}pipe}\PY{p}{,} \PY{n}{param\PYZus{}grid}\PY{o}{=}\PY{n}{grid\PYZus{}params}\PY{p}{,} \PY{n}{cv}\PY{o}{=}\PY{l+m+mi}{5}\PY{p}{,} \PY{n}{n\PYZus{}jobs}\PY{o}{=}\PY{o}{\PYZhy{}}\PY{l+m+mi}{1}\PY{p}{)}
\end{Verbatim}


    \begin{Verbatim}[commandchars=\\\{\}]
{\color{incolor}In [{\color{incolor}110}]:} \PY{c+c1}{\PYZsh{}Code task 25\PYZsh{}}
          \PY{c+c1}{\PYZsh{}Now call the `GridSearchCV`\PYZsq{}s `fit()` method with `X\PYZus{}train` and `y\PYZus{}train` as arguments}
          \PY{c+c1}{\PYZsh{}to actually start the grid search. This may take a minute or two.}
          \PY{n}{rf\PYZus{}grid\PYZus{}cv}\PY{o}{.}\PY{n}{fit}\PY{p}{(}\PY{n}{X\PYZus{}train}\PY{p}{,} \PY{n}{y\PYZus{}train}\PY{p}{)}
\end{Verbatim}


\begin{Verbatim}[commandchars=\\\{\}]
{\color{outcolor}Out[{\color{outcolor}110}]:} GridSearchCV(cv=5,
                       estimator=Pipeline(steps=[('simpleimputer',
                                                  SimpleImputer(strategy='median')),
                                                 ('standardscaler', StandardScaler()),
                                                 ('randomforestregressor',
                                                  RandomForestRegressor(random\_state=47))]),
                       n\_jobs=-1,
                       param\_grid=\{'randomforestregressor\_\_n\_estimators': [10, 12, 16, 20,
                                                                           26, 33, 42, 54,
                                                                           69, 88, 112,
                                                                           143, 183, 233,
                                                                           297, 379, 483,
                                                                           615, 784,
                                                                           1000],
                                   'simpleimputer\_\_strategy': ['mean', 'median'],
                                   'standardscaler': [StandardScaler(), None]\})
\end{Verbatim}
            
    \begin{Verbatim}[commandchars=\\\{\}]
{\color{incolor}In [{\color{incolor}111}]:} \PY{c+c1}{\PYZsh{}Code task 26\PYZsh{}}
          \PY{c+c1}{\PYZsh{}Print the best params (`best\PYZus{}params\PYZus{}` attribute) from the grid search}
          \PY{n}{rf\PYZus{}grid\PYZus{}cv}\PY{o}{.}\PY{n}{best\PYZus{}params\PYZus{}}
\end{Verbatim}


\begin{Verbatim}[commandchars=\\\{\}]
{\color{outcolor}Out[{\color{outcolor}111}]:} \{'randomforestregressor\_\_n\_estimators': 69,
           'simpleimputer\_\_strategy': 'median',
           'standardscaler': None\}
\end{Verbatim}
            
    It looks like imputing with the median helps, but scaling the features
doesn't.

    \begin{Verbatim}[commandchars=\\\{\}]
{\color{incolor}In [{\color{incolor}112}]:} \PY{n}{rf\PYZus{}best\PYZus{}cv\PYZus{}results} \PY{o}{=} \PY{n}{cross\PYZus{}validate}\PY{p}{(}\PY{n}{rf\PYZus{}grid\PYZus{}cv}\PY{o}{.}\PY{n}{best\PYZus{}estimator\PYZus{}}\PY{p}{,} \PY{n}{X\PYZus{}train}\PY{p}{,} \PY{n}{y\PYZus{}train}\PY{p}{,} \PY{n}{cv}\PY{o}{=}\PY{l+m+mi}{5}\PY{p}{)}
          \PY{n}{rf\PYZus{}best\PYZus{}scores} \PY{o}{=} \PY{n}{rf\PYZus{}best\PYZus{}cv\PYZus{}results}\PY{p}{[}\PY{l+s+s1}{\PYZsq{}}\PY{l+s+s1}{test\PYZus{}score}\PY{l+s+s1}{\PYZsq{}}\PY{p}{]}
          \PY{n}{rf\PYZus{}best\PYZus{}scores}
\end{Verbatim}


\begin{Verbatim}[commandchars=\\\{\}]
{\color{outcolor}Out[{\color{outcolor}112}]:} array([0.68699557, 0.81671338, 0.76934331, 0.62158696, 0.66858528])
\end{Verbatim}
            
    \begin{Verbatim}[commandchars=\\\{\}]
{\color{incolor}In [{\color{incolor}113}]:} \PY{n}{np}\PY{o}{.}\PY{n}{mean}\PY{p}{(}\PY{n}{rf\PYZus{}best\PYZus{}scores}\PY{p}{)}\PY{p}{,} \PY{n}{np}\PY{o}{.}\PY{n}{std}\PY{p}{(}\PY{n}{rf\PYZus{}best\PYZus{}scores}\PY{p}{)}
\end{Verbatim}


\begin{Verbatim}[commandchars=\\\{\}]
{\color{outcolor}Out[{\color{outcolor}113}]:} (0.7126449010338891, 0.07061960547565885)
\end{Verbatim}
            
    You've marginally improved upon the default CV results. Random forest
has many more hyperparameters you could tune, but we won't dive into
that here.

    \begin{Verbatim}[commandchars=\\\{\}]
{\color{incolor}In [{\color{incolor}114}]:} \PY{c+c1}{\PYZsh{}Code task 27\PYZsh{}}
          \PY{c+c1}{\PYZsh{}Plot a barplot of the random forest\PYZsq{}s feature importances,}
          \PY{c+c1}{\PYZsh{}assigning the `feature\PYZus{}importances\PYZus{}` attribute of }
          \PY{c+c1}{\PYZsh{}`rf\PYZus{}grid\PYZus{}cv.best\PYZus{}estimator\PYZus{}.named\PYZus{}steps.randomforestregressor` to the name `imps` to then}
          \PY{c+c1}{\PYZsh{}create a pandas Series object of the feature importances, with the index given by the}
          \PY{c+c1}{\PYZsh{}training data column names, sorting the values in descending order}
          \PY{n}{plt}\PY{o}{.}\PY{n}{subplots}\PY{p}{(}\PY{n}{figsize}\PY{o}{=}\PY{p}{(}\PY{l+m+mi}{10}\PY{p}{,} \PY{l+m+mi}{5}\PY{p}{)}\PY{p}{)}
          \PY{n}{imps} \PY{o}{=} \PY{n}{rf\PYZus{}grid\PYZus{}cv}\PY{o}{.}\PY{n}{best\PYZus{}estimator\PYZus{}}\PY{o}{.}\PY{n}{named\PYZus{}steps}\PY{o}{.}\PY{n}{randomforestregressor}\PY{o}{.}\PY{n}{feature\PYZus{}importances\PYZus{}}
          \PY{n}{rf\PYZus{}feat\PYZus{}imps} \PY{o}{=} \PY{n}{pd}\PY{o}{.}\PY{n}{Series}\PY{p}{(}\PY{n}{imps}\PY{p}{,} \PY{n}{index}\PY{o}{=}\PY{n}{X\PYZus{}train}\PY{o}{.}\PY{n}{columns}\PY{p}{)}\PY{o}{.}\PY{n}{sort\PYZus{}values}\PY{p}{(}\PY{n}{ascending}\PY{o}{=}\PY{k+kc}{False}\PY{p}{)}
          \PY{n}{rf\PYZus{}feat\PYZus{}imps}\PY{o}{.}\PY{n}{plot}\PY{p}{(}\PY{n}{kind}\PY{o}{=}\PY{l+s+s1}{\PYZsq{}}\PY{l+s+s1}{bar}\PY{l+s+s1}{\PYZsq{}}\PY{p}{)}
          \PY{n}{plt}\PY{o}{.}\PY{n}{xlabel}\PY{p}{(}\PY{l+s+s1}{\PYZsq{}}\PY{l+s+s1}{features}\PY{l+s+s1}{\PYZsq{}}\PY{p}{)}
          \PY{n}{plt}\PY{o}{.}\PY{n}{ylabel}\PY{p}{(}\PY{l+s+s1}{\PYZsq{}}\PY{l+s+s1}{importance}\PY{l+s+s1}{\PYZsq{}}\PY{p}{)}
          \PY{n}{plt}\PY{o}{.}\PY{n}{title}\PY{p}{(}\PY{l+s+s1}{\PYZsq{}}\PY{l+s+s1}{Best random forest regressor feature importances}\PY{l+s+s1}{\PYZsq{}}\PY{p}{)}\PY{p}{;}
\end{Verbatim}


    \begin{center}
    \adjustimage{max size={0.9\linewidth}{0.9\paperheight}}{output_204_0.png}
    \end{center}
    { \hspace*{\fill} \\}
    
    Encouragingly, the dominant top four features are in common with your
linear model: * fastQuads * Runs * Snow Making\_ac * vertical\_drop

    \subsection{4.11 Final Model Selection}\label{final-model-selection}

    Time to select your final model to use for further business modeling! It
would be good to revisit the above model selection; there is undoubtedly
more that could be done to explore possible hyperparameters. It would
also be worthwhile to investigate removing the least useful features.
Gathering or calculating, and storing, features adds business cost and
dependencies, so if features genuinely are not needed they should be
removed. Building a simpler model with fewer features can also have the
advantage of being easier to sell (and/or explain) to stakeholders.
Certainly there seem to be four strong features here and so a model
using only those would probably work well. However, you want to explore
some different scenarios where other features vary so keep the fuller
model for now. The business is waiting for this model and you have
something that you have confidence in to be much better than guessing
with the average price.

Or, rather, you have two "somethings". You built a best linear model and
a best random forest model. You need to finally choose between them. You
can calculate the mean absolute error using cross-validation. Although
\texttt{cross-validate} defaults to the \(R^2\)
\href{https://scikit-learn.org/stable/modules/model_evaluation.html\#scoring}{metric
for scoring} regression, you can specify the mean absolute error as an
alternative via the \texttt{scoring} parameter.

    \subsubsection{4.11.1 Linear regression model
performance}\label{linear-regression-model-performance}

    \begin{Verbatim}[commandchars=\\\{\}]
{\color{incolor}In [{\color{incolor}115}]:} \PY{c+c1}{\PYZsh{} \PYZsq{}neg\PYZus{}mean\PYZus{}absolute\PYZus{}error\PYZsq{} uses the (negative of) the mean absolute error}
          \PY{n}{lr\PYZus{}neg\PYZus{}mae} \PY{o}{=} \PY{n}{cross\PYZus{}validate}\PY{p}{(}\PY{n}{lr\PYZus{}grid\PYZus{}cv}\PY{o}{.}\PY{n}{best\PYZus{}estimator\PYZus{}}\PY{p}{,} \PY{n}{X\PYZus{}train}\PY{p}{,} \PY{n}{y\PYZus{}train}\PY{p}{,} 
                                      \PY{n}{scoring}\PY{o}{=}\PY{l+s+s1}{\PYZsq{}}\PY{l+s+s1}{neg\PYZus{}mean\PYZus{}absolute\PYZus{}error}\PY{l+s+s1}{\PYZsq{}}\PY{p}{,} \PY{n}{cv}\PY{o}{=}\PY{l+m+mi}{5}\PY{p}{,} \PY{n}{n\PYZus{}jobs}\PY{o}{=}\PY{o}{\PYZhy{}}\PY{l+m+mi}{1}\PY{p}{)}
\end{Verbatim}


    \begin{Verbatim}[commandchars=\\\{\}]
{\color{incolor}In [{\color{incolor}116}]:} \PY{n}{lr\PYZus{}mae\PYZus{}mean} \PY{o}{=} \PY{n}{np}\PY{o}{.}\PY{n}{mean}\PY{p}{(}\PY{o}{\PYZhy{}}\PY{l+m+mi}{1} \PY{o}{*} \PY{n}{lr\PYZus{}neg\PYZus{}mae}\PY{p}{[}\PY{l+s+s1}{\PYZsq{}}\PY{l+s+s1}{test\PYZus{}score}\PY{l+s+s1}{\PYZsq{}}\PY{p}{]}\PY{p}{)}
          \PY{n}{lr\PYZus{}mae\PYZus{}std} \PY{o}{=} \PY{n}{np}\PY{o}{.}\PY{n}{std}\PY{p}{(}\PY{o}{\PYZhy{}}\PY{l+m+mi}{1} \PY{o}{*} \PY{n}{lr\PYZus{}neg\PYZus{}mae}\PY{p}{[}\PY{l+s+s1}{\PYZsq{}}\PY{l+s+s1}{test\PYZus{}score}\PY{l+s+s1}{\PYZsq{}}\PY{p}{]}\PY{p}{)}
          \PY{n}{lr\PYZus{}mae\PYZus{}mean}\PY{p}{,} \PY{n}{lr\PYZus{}mae\PYZus{}std}
\end{Verbatim}


\begin{Verbatim}[commandchars=\\\{\}]
{\color{outcolor}Out[{\color{outcolor}116}]:} (10.499032338015294, 1.622060897679966)
\end{Verbatim}
            
    \begin{Verbatim}[commandchars=\\\{\}]
{\color{incolor}In [{\color{incolor}117}]:} \PY{n}{mean\PYZus{}absolute\PYZus{}error}\PY{p}{(}\PY{n}{y\PYZus{}test}\PY{p}{,} \PY{n}{lr\PYZus{}grid\PYZus{}cv}\PY{o}{.}\PY{n}{best\PYZus{}estimator\PYZus{}}\PY{o}{.}\PY{n}{predict}\PY{p}{(}\PY{n}{X\PYZus{}test}\PY{p}{)}\PY{p}{)}
\end{Verbatim}


\begin{Verbatim}[commandchars=\\\{\}]
{\color{outcolor}Out[{\color{outcolor}117}]:} 11.793465668669324
\end{Verbatim}
            
    \subsubsection{4.11.2 Random forest regression model
performance}\label{random-forest-regression-model-performance}

    \begin{Verbatim}[commandchars=\\\{\}]
{\color{incolor}In [{\color{incolor}118}]:} \PY{n}{rf\PYZus{}neg\PYZus{}mae} \PY{o}{=} \PY{n}{cross\PYZus{}validate}\PY{p}{(}\PY{n}{rf\PYZus{}grid\PYZus{}cv}\PY{o}{.}\PY{n}{best\PYZus{}estimator\PYZus{}}\PY{p}{,} \PY{n}{X\PYZus{}train}\PY{p}{,} \PY{n}{y\PYZus{}train}\PY{p}{,} 
                                      \PY{n}{scoring}\PY{o}{=}\PY{l+s+s1}{\PYZsq{}}\PY{l+s+s1}{neg\PYZus{}mean\PYZus{}absolute\PYZus{}error}\PY{l+s+s1}{\PYZsq{}}\PY{p}{,} \PY{n}{cv}\PY{o}{=}\PY{l+m+mi}{5}\PY{p}{,} \PY{n}{n\PYZus{}jobs}\PY{o}{=}\PY{o}{\PYZhy{}}\PY{l+m+mi}{1}\PY{p}{)}
\end{Verbatim}


    \begin{Verbatim}[commandchars=\\\{\}]
{\color{incolor}In [{\color{incolor}119}]:} \PY{n}{rf\PYZus{}mae\PYZus{}mean} \PY{o}{=} \PY{n}{np}\PY{o}{.}\PY{n}{mean}\PY{p}{(}\PY{o}{\PYZhy{}}\PY{l+m+mi}{1} \PY{o}{*} \PY{n}{rf\PYZus{}neg\PYZus{}mae}\PY{p}{[}\PY{l+s+s1}{\PYZsq{}}\PY{l+s+s1}{test\PYZus{}score}\PY{l+s+s1}{\PYZsq{}}\PY{p}{]}\PY{p}{)}
          \PY{n}{rf\PYZus{}mae\PYZus{}std} \PY{o}{=} \PY{n}{np}\PY{o}{.}\PY{n}{std}\PY{p}{(}\PY{o}{\PYZhy{}}\PY{l+m+mi}{1} \PY{o}{*} \PY{n}{rf\PYZus{}neg\PYZus{}mae}\PY{p}{[}\PY{l+s+s1}{\PYZsq{}}\PY{l+s+s1}{test\PYZus{}score}\PY{l+s+s1}{\PYZsq{}}\PY{p}{]}\PY{p}{)}
          \PY{n}{rf\PYZus{}mae\PYZus{}mean}\PY{p}{,} \PY{n}{rf\PYZus{}mae\PYZus{}std}
\end{Verbatim}


\begin{Verbatim}[commandchars=\\\{\}]
{\color{outcolor}Out[{\color{outcolor}119}]:} (9.651183457528996, 1.4940374628899329)
\end{Verbatim}
            
    \begin{Verbatim}[commandchars=\\\{\}]
{\color{incolor}In [{\color{incolor}120}]:} \PY{n}{mean\PYZus{}absolute\PYZus{}error}\PY{p}{(}\PY{n}{y\PYZus{}test}\PY{p}{,} \PY{n}{rf\PYZus{}grid\PYZus{}cv}\PY{o}{.}\PY{n}{best\PYZus{}estimator\PYZus{}}\PY{o}{.}\PY{n}{predict}\PY{p}{(}\PY{n}{X\PYZus{}test}\PY{p}{)}\PY{p}{)}
\end{Verbatim}


\begin{Verbatim}[commandchars=\\\{\}]
{\color{outcolor}Out[{\color{outcolor}120}]:} 9.5253256504278
\end{Verbatim}
            
    \subsubsection{4.11.3 Conclusion}\label{conclusion}

    The random forest model has a lower cross-validation mean absolute error
by almost \textbackslash{}\$1. It also exhibits less variability.
Verifying performance on the test set produces performance consistent
with the cross-validation results.

    \subsection{4.12 Data quantity
assessment}\label{data-quantity-assessment}

    Finally, you need to advise the business whether it needs to undertake
further data collection. Would more data be useful? We're often led to
believe more data is always good, but gathering data invariably has a
cost associated with it. Assess this trade off by seeing how performance
varies with differing data set sizes. The \texttt{learning\_curve}
function does this conveniently.

    \begin{Verbatim}[commandchars=\\\{\}]
{\color{incolor}In [{\color{incolor}121}]:} \PY{n}{fractions} \PY{o}{=} \PY{p}{[}\PY{o}{.}\PY{l+m+mi}{2}\PY{p}{,} \PY{o}{.}\PY{l+m+mi}{25}\PY{p}{,} \PY{o}{.}\PY{l+m+mi}{3}\PY{p}{,} \PY{o}{.}\PY{l+m+mi}{35}\PY{p}{,} \PY{o}{.}\PY{l+m+mi}{4}\PY{p}{,} \PY{o}{.}\PY{l+m+mi}{45}\PY{p}{,} \PY{o}{.}\PY{l+m+mi}{5}\PY{p}{,} \PY{o}{.}\PY{l+m+mi}{6}\PY{p}{,} \PY{o}{.}\PY{l+m+mi}{75}\PY{p}{,} \PY{o}{.}\PY{l+m+mi}{8}\PY{p}{,} \PY{l+m+mf}{1.0}\PY{p}{]}
          \PY{n}{train\PYZus{}size}\PY{p}{,} \PY{n}{train\PYZus{}scores}\PY{p}{,} \PY{n}{test\PYZus{}scores} \PY{o}{=} \PY{n}{learning\PYZus{}curve}\PY{p}{(}\PY{n}{pipe}\PY{p}{,} \PY{n}{X\PYZus{}train}\PY{p}{,} \PY{n}{y\PYZus{}train}\PY{p}{,} \PY{n}{train\PYZus{}sizes}\PY{o}{=}\PY{n}{fractions}\PY{p}{)}
          \PY{n}{train\PYZus{}scores\PYZus{}mean} \PY{o}{=} \PY{n}{np}\PY{o}{.}\PY{n}{mean}\PY{p}{(}\PY{n}{train\PYZus{}scores}\PY{p}{,} \PY{n}{axis}\PY{o}{=}\PY{l+m+mi}{1}\PY{p}{)}
          \PY{n}{train\PYZus{}scores\PYZus{}std} \PY{o}{=} \PY{n}{np}\PY{o}{.}\PY{n}{std}\PY{p}{(}\PY{n}{train\PYZus{}scores}\PY{p}{,} \PY{n}{axis}\PY{o}{=}\PY{l+m+mi}{1}\PY{p}{)}
          \PY{n}{test\PYZus{}scores\PYZus{}mean} \PY{o}{=} \PY{n}{np}\PY{o}{.}\PY{n}{mean}\PY{p}{(}\PY{n}{test\PYZus{}scores}\PY{p}{,} \PY{n}{axis}\PY{o}{=}\PY{l+m+mi}{1}\PY{p}{)}
          \PY{n}{test\PYZus{}scores\PYZus{}std} \PY{o}{=} \PY{n}{np}\PY{o}{.}\PY{n}{std}\PY{p}{(}\PY{n}{test\PYZus{}scores}\PY{p}{,} \PY{n}{axis}\PY{o}{=}\PY{l+m+mi}{1}\PY{p}{)}
\end{Verbatim}


    \begin{Verbatim}[commandchars=\\\{\}]
{\color{incolor}In [{\color{incolor}122}]:} \PY{n}{plt}\PY{o}{.}\PY{n}{subplots}\PY{p}{(}\PY{n}{figsize}\PY{o}{=}\PY{p}{(}\PY{l+m+mi}{10}\PY{p}{,} \PY{l+m+mi}{5}\PY{p}{)}\PY{p}{)}
          \PY{n}{plt}\PY{o}{.}\PY{n}{errorbar}\PY{p}{(}\PY{n}{train\PYZus{}size}\PY{p}{,} \PY{n}{test\PYZus{}scores\PYZus{}mean}\PY{p}{,} \PY{n}{yerr}\PY{o}{=}\PY{n}{test\PYZus{}scores\PYZus{}std}\PY{p}{)}
          \PY{n}{plt}\PY{o}{.}\PY{n}{xlabel}\PY{p}{(}\PY{l+s+s1}{\PYZsq{}}\PY{l+s+s1}{Training set size}\PY{l+s+s1}{\PYZsq{}}\PY{p}{)}
          \PY{n}{plt}\PY{o}{.}\PY{n}{ylabel}\PY{p}{(}\PY{l+s+s1}{\PYZsq{}}\PY{l+s+s1}{CV scores}\PY{l+s+s1}{\PYZsq{}}\PY{p}{)}
          \PY{n}{plt}\PY{o}{.}\PY{n}{title}\PY{p}{(}\PY{l+s+s1}{\PYZsq{}}\PY{l+s+s1}{Cross\PYZhy{}validation score as training set size increases}\PY{l+s+s1}{\PYZsq{}}\PY{p}{)}\PY{p}{;}
\end{Verbatim}


    \begin{center}
    \adjustimage{max size={0.9\linewidth}{0.9\paperheight}}{output_221_0.png}
    \end{center}
    { \hspace*{\fill} \\}
    
    This shows that you seem to have plenty of data. There's an initial
rapid improvement in model scores as one would expect, but it's
essentially levelled off by around a sample size of 40-50.

    \subsection{4.13 Save best model object from
pipeline}\label{save-best-model-object-from-pipeline}

    \begin{Verbatim}[commandchars=\\\{\}]
{\color{incolor}In [{\color{incolor}123}]:} \PY{c+c1}{\PYZsh{}Code task 28\PYZsh{}}
          \PY{c+c1}{\PYZsh{}This may not be \PYZdq{}production grade ML deployment\PYZdq{} practice, but adding some basic}
          \PY{c+c1}{\PYZsh{}information to your saved models can save your bacon in development.}
          \PY{c+c1}{\PYZsh{}Just what version model have you just loaded to reuse? What version of `sklearn`}
          \PY{c+c1}{\PYZsh{}created it? When did you make it?}
          \PY{c+c1}{\PYZsh{}Assign the pandas version number (`pd.\PYZus{}\PYZus{}version\PYZus{}\PYZus{}`) to the `pandas\PYZus{}version` attribute,}
          \PY{c+c1}{\PYZsh{}the numpy version (`np.\PYZus{}\PYZus{}version\PYZus{}\PYZus{}`) to the `numpy\PYZus{}version` attribute,}
          \PY{c+c1}{\PYZsh{}the sklearn version (`sklearn\PYZus{}version`) to the `sklearn\PYZus{}version` attribute,}
          \PY{c+c1}{\PYZsh{}and the current datetime (`datetime.datetime.now()`) to the `build\PYZus{}datetime` attribute}
          \PY{c+c1}{\PYZsh{}Let\PYZsq{}s call this model version \PYZsq{}1.0\PYZsq{}}
          \PY{n}{best\PYZus{}model} \PY{o}{=} \PY{n}{rf\PYZus{}grid\PYZus{}cv}\PY{o}{.}\PY{n}{best\PYZus{}estimator\PYZus{}}
          \PY{n}{best\PYZus{}model}\PY{o}{.}\PY{n}{version} \PY{o}{=} \PY{l+s+s1}{\PYZsq{}}\PY{l+s+s1}{1.0}\PY{l+s+s1}{\PYZsq{}}
          \PY{n}{best\PYZus{}model}\PY{o}{.}\PY{n}{pandas\PYZus{}version} \PY{o}{=} \PY{n}{pd}\PY{o}{.}\PY{n}{\PYZus{}\PYZus{}version\PYZus{}\PYZus{}}
          \PY{n}{best\PYZus{}model}\PY{o}{.}\PY{n}{numpy\PYZus{}version} \PY{o}{=} \PY{n}{np}\PY{o}{.}\PY{n}{\PYZus{}\PYZus{}version\PYZus{}\PYZus{}}
          \PY{n}{best\PYZus{}model}\PY{o}{.}\PY{n}{sklearn\PYZus{}version} \PY{o}{=} \PY{n}{sklearn\PYZus{}version}
          \PY{n}{best\PYZus{}model}\PY{o}{.}\PY{n}{X\PYZus{}columns} \PY{o}{=} \PY{p}{[}\PY{n}{col} \PY{k}{for} \PY{n}{col} \PY{o+ow}{in} \PY{n}{X\PYZus{}train}\PY{o}{.}\PY{n}{columns}\PY{p}{]}
          \PY{n}{best\PYZus{}model}\PY{o}{.}\PY{n}{build\PYZus{}datetime} \PY{o}{=} \PY{n}{datetime}\PY{o}{.}\PY{n}{datetime}\PY{o}{.}\PY{n}{now}\PY{p}{(}\PY{p}{)}
\end{Verbatim}


    \begin{Verbatim}[commandchars=\\\{\}]
{\color{incolor}In [{\color{incolor}124}]:} \PY{c+c1}{\PYZsh{} save the model}
          
          \PY{n}{modelpath} \PY{o}{=} \PY{l+s+s1}{\PYZsq{}}\PY{l+s+s1}{../models}\PY{l+s+s1}{\PYZsq{}}
          \PY{n}{save\PYZus{}file}\PY{p}{(}\PY{n}{best\PYZus{}model}\PY{p}{,} \PY{l+s+s1}{\PYZsq{}}\PY{l+s+s1}{ski\PYZus{}resort\PYZus{}pricing\PYZus{}model.pkl}\PY{l+s+s1}{\PYZsq{}}\PY{p}{,} \PY{n}{modelpath}\PY{p}{)}
\end{Verbatim}


    \begin{Verbatim}[commandchars=\\\{\}]
Directory ../models was created.
Writing file.  "../models\textbackslash{}ski\_resort\_pricing\_model.pkl"

    \end{Verbatim}

    \subsection{4.14 Summary}\label{summary}

    \textbf{Q: 1} Write a summary of the work in this notebook. Capture the
fact that you gained a baseline idea of performance by simply taking the
average price and how well that did. Then highlight that you built a
linear model and the features that found. Comment on the estimate of its
performance from cross-validation and whether its performance on the
test split was consistent with this estimate. Also highlight that a
random forest regressor was tried, what preprocessing steps were found
to be best, and again what its estimated performance via
cross-validation was and whether its performance on the test set was
consistent with that. State which model you have decided to use going
forwards and why. This summary should provide a quick overview for
someone wanting to know quickly why the given model was chosen for the
next part of the business problem to help guide important business
decisions.

    \textbf{A: 1} Your answer here


    % Add a bibliography block to the postdoc
    
    
    
    \end{document}
